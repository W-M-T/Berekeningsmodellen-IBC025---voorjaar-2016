% vim: set spelllang=nl:
\subsection{Toestanden}
\label{sec:def:state}

Een toestand $s\in\State$ van Smurf bevat zowel een stack als een variable
store. Hieruit volgt de voor de hand liggende definitie voor $\State$,
$$\State \isdef \Stack{\String} \times (\String \to \String),$$
waarbij we $s=(\stk,\str)\in\State$ lezen als de toestand $s$ met stack $\stk$
en variable store $\str$.

De variable store is een totale functie. Initieel zijn alle waardes in de store
$\lambda$ in overeenstemming met de documentatie \cite{safalra}:

\begin{quote}
	Any string can be used as a variable name, including the empty string. All
	variable values are initially set to the empty string.
\end{quote}

Een toewijzing van $\val$ aan $\var$ noteren we als $\var\mapsto\val$, zodat we
bijvoorbeeld $\{\texttt{x}\mapsto\texttt{hello world}\}$ noteren voor de store
die \texttt{x} naar ``hello world'' stuurt. We laten de strings die naar
$\lambda$ sturen normaal gesproken weg. De initiële store schrijven we dus als
$\emptystore$.

Om de waarde van een key $k$ uit store $\str$ te halen gebruiken we simpelweg
$\str~k$. Vervolgens definiëren we $\putop:\SynString \times \SynString \times
(\String\to\String) \to (\String\to\String)$ die gegeven een variabelenaam, een
waarde en een oude store een nieuwe store oplevert:

$$
	\put{\var}{\val}{\str} k =
		\begin{cases}
			\val   & \text{als $k=\var$} \\
			\str~k & \text{als $k\ne\var$}
		\end{cases}
$$

