% vim: set spelllang=nl:
\subsection{Commando's}
\label{sec:intro:commands}
We zullen nu kort op informele wijze de verschillende commando's in Smurf
beschrijven. We geven telkens de notatie in Smurf syntax en een leesbaarder
alternatief dat we hieronder zullen gebruiken. Wanneer een commando iets met
elementen op de stack doet, worden die elementen altijd verwijderd.
Merk op dat de commando's niet exact hetzelfde zijn als in de
voorbeeldprogramma's en de taalspecificatie. We gebruiken een leesbaardere
variant op de taal zodat het overzichtelijker is om eigenschappen ervan te
bespreken. Alle commando's betekenen nog steeds hetzelfde.

\begin{description}[style=nextline,font=\normalfont]
	\item[\smurfinline{"..."} of $\StmPush~\texttt{...}$]
		waarbij `\texttt{...}' een string is. Zet de string \texttt{...} op de
		stack.
	\item[\smurfinline{+} of $\StmCat$]
		Concateneert de bovenste twee strings (laagste eerst) op de stack en zet
		het resultaat op de stack.
	\item[\smurfinline{i} of $\StmInput$]
		Plaatst een string van `user input' op de stack. Hierbij wordt
		\texttt{\textbackslash} gebruikt om LF-karakters, dubbele aanhalingstekens
		en backslashes te escapen. Het is ook mogelijk LF-karakters in de string te
		gebruiken. Afhankelijk van de inputmethode is dit al dan niet mogelijk ---
		in deze specificatie abstraheren we van inputmethodes.
	\item[\smurfinline{o} of $\StmOutput$]
		Stuurt het bovenste element van de stack naar `de output'.
	\item[\smurfinline{p} of $\StmPut$]
		Hierbij wordt ervoor gezorgd dat de waarde van de variabelenaam bovenop de
		stack verwijst naar de string die daaronder staat.
    \item[\smurfinline{g} of $\StmGet$]
		Zoekt de variabele op met als naam de bovenste waarde van de stack en zet de waarde van die variabele bovenop de stack.
	\item[\smurfinline{h} of $\StmHead$]
		Vervang de string bovenop de stack met zijn eerste karakter.
	\item[\smurfinline{t} of $\StmTail$]
		Zet alles behalve de head van de string bovenop de stack op de stack.
	\item[\smurfinline{q} of $\StmQuotify$]
		Manipuleert de string bovenop de stack zodat die als $\StmPush$-commando in
		een Smurfprogramma kan worden gebruikt: escapet LF-karakters, dubbele
		aanhalingstekens en back\-slashes met een backslash, en plaatst dubbele
		aanhalingstekens om de hele string. Het resultaat wordt op de stack gezet.
	\item[\smurfinline{x} of $\StmExec$]
		Voert de string bovenop de stack uit als Smurfprogramma. Het programma
		begint met een schone toestand, d.w.z. met een lege stack en variable
		store. Dit commando maakt condities, recursie en iteratie mogelijk.
\end{description}

