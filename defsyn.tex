% vim: set spelllang=nl:
\subsection{Syntax}
\label{sec:def:syn}
We definiëren de volgende syntax:
\setlength{\grammarindent}{5em}
\begin{grammar}
	<Pgm> ::= <Stm>:<Pgm> | $\lambda$ %todo over de : moet uitleg komen, waarom
	%we die introduceren

	<Stm> ::= `Push' <String>
		\alt `Cat' | `Head' | `Tail' | `Quotify'
		\alt `Put' | `Get'
		\alt `Input' | `Output'
		\alt `Exec'

	<String> ::= <Char><String> | $\lambda$
\end{grammar}

Een karakter, $\SynChar$, is een symbool uit de ASCII tabel.

Programma's zijn lijsten van statements. Merk op dat compositie van statements
hier expliciet is door middel van de \lit{:}, waar compositie in de
oorspronkelijke versie van de syntax impliciet was.
