% vim: set spelllang=nl:
\subsection{\texttt{Output}}

\begin{quote}
	o - Output the string at the top of the stack
\end{quote}

Net als bij het inputcommando gaan we op een abstracte wijze met de output om.
We houden gedurende het hele programma een stack van strings, genaamd $\Output$
bij waar het programma zijn output naar wegschrijft.

Dit geeft de volgende regel:

$$
\prooftree
        \trans
        {\pgm}{\ip}{(\stk',\str)}
        	{\ip'}{[\op:\Nil]}{\st}
	\justifies
        \trans
        {\StmOutput:\pgm}{\ip}{(\stk,\str)}
            {\ip'}{[\op:[s:\Nil]]}{\st}
	\using{\routputns}
	\qquad
	\text{met $(s,\stk') = \pop{\stk}$.}
\endprooftree
$$

Merk op dat eenzelfde regel waar $s$ niet achteraan maar vooraan zou komen te
staan, even geldig is. Geen van beide opties is beter dan de ander omdat we
geen aannames doen over hoe de $\Output$-stack wordt verwerkt.

