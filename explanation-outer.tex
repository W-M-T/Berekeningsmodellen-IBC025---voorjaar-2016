% vim: set spelllang=nl:
\subsubsection{Het buitenste programma}
Het buitenste programma controleert of we te maken hebben met de lege string
als inputstring. Als dit het geval is geeft het de lege string als output.
Anders wordt het recursieve subprogramma gebruikt, dat alleen voor strings van
lengtes groter dan nul gedefini\"{e}erd is.

Omdat Smurf geen conditionals kent, moeten we hier op een slimme manier omgaan
met het feit dat waarden ook als variabelenaam kunnen worden gebruikt. Dit is
vergelijkbaar met \autoref{exmp:headortail}. Eerst schrijven we het recursieve
programma dat we moeten aanroepen als de inputstring niet leeg is naar de
variabele met als naam de inputstring zelf. Hierna schrijven we het programma
dat moet worden uitgevoerd als de string leeg is (het outputten van de lege
string) naar de variabele met als naam de lege string. Als de inputstring leeg
is overschrijven we daar dus het recursieve programma dat we eerst aan die
variabele hadden toegewezen. Als laatste voeren we het programma uit dat op de
variabele met naam de inputstring staat.

Het buitenste programma ziet er als volgt uit:
\begin{center}
\makebox{
\parbox{35mm}{
\begin{smurf}
\footnotesize
i "input" p\\
"input" g\\
"input" g q\\
"\textbackslash{}"\textbackslash{}""\emph{subprogramma-string}\\
+\\
"input" g p\\
"\textbackslash{}"\textbackslash{}"o" "" p\\
g x
\end{smurf}
}

\parbox{8mm}{
\begin{smurf}
\footnotesize
(1)\\
(2)\\
(3)\\
.\\
.\\
(4)\\
(5)\\
(6)
\end{smurf}
}}
\end{center}

\begin{description}
\item \textbf{(1)} Allereerst wordt de inputstring op de stack gezet en naar de
variabele met naam "input" geschreven. De stack is nu weer leeg.
\item \textbf{(2)}
Nu zetten we de inputwaarde weer op de stack voor later gebruik. De reden
hiervoor volgt straks.
\item \textbf{(3)} Hierna halen we deze inputwaarde voor de
tweede keer op, quotifyen we hem en voegen we het toe aan het begin van de
subprogramma-string. Dit kun je zien als het doorgeven van een argument. De
reden dat we de input quotifyen is dat dit ervoor zorgt dat alle karakters in
de input behouden blijven (ook die met speciale betekenis zoals
"\textbackslash{}") wanneer het subprogramma wordt uitgevoerd. Het subprogramma
staat zelf al gequotifyed hier.  Op dit moment staat het grote programma met de
input eraan voorgevoegd op de stack, met als element eronder de inputwaarde die
we zo nodig hebben.
\item \textbf{(4)} Nu halen we opnieuw de waarde van de input op
uit de variabele waarnaar we het eerst hadden weggeschreven en gebruiken we dit
als een variabelenaam om het subprogramma naar weg te schrijven. Merk op dat we
onze opslag van de inputstring verliezen als die toevallig de waarde "input"
had, omdat we die variabele dan overschrijven. Dit is waarom we deze waarde van
tevoren een keer extra op de stack hebben gezet. Zo kunnen we hem bij (6) nog
gebruiken.
\item \textbf{(5)} Nu zetten we het gequotifyde programma "output de lege
string" op het register met naam de lege string. Ongequotifyed ziet het er zo
uit: \texttt{""o}
\item \textbf{(6)} Als laatste halen we het programma op van de
variabele met als naam de inputstring die we in (2) op de stack hadden gezet en
voeren we dit programma uit.
\end{description}