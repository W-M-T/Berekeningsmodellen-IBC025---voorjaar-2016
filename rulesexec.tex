% vim: set spelllang=nl:
\subsection{\texttt{Exec}}

\begin{quote}
	x - Executes the string at the top of the stack as a Smurf program. The stack
	and variable store are erased.
\end{quote}

We halen een string van de stack en gebruiken $\parsepgmop$ om dit in een
programma om te zetten. Dit wordt het nieuwe programma om uitgevoerd te worden.
Als de $\stk$ leeg is is deze regel niet toepasbaar, omdat $\pop\stk$ dan niet
gedefinieerd is. Ook is deze regel niet toepasbaar als de gepopte string zelf
geen geldig Smurf-programma is, omdat $\parsepgmop$ dan niet gedefinieerd is.

$$
\begin{prooftree}
	\trans
		{\pgm'}{\ip}{(\Nil, \emptystore)}
		{\ip'}{\op}{\st}
	\justifies
	\trans
		{\StmExec:\pgm}{\ip}{(\stk,\str)}
		{\ip'}{\op}{\st}
	\using{\rexecns}
	\qquad
	\text{met\enspace
		\parbox{36mm}{$(\var,\stk') = \pop{\stk}$,\\
			$\pgm' = \parsepgm{\var'}$.}
	}
\end{prooftree}
$$

\medskip
$\parsepgmop$ definiëren we als volgt, met een hulpfunctie $\parsestrop$:

$$
	\parsepgm s =
		\begin{cases}
			\lambda                   & \text{als $s=\lambda$}\\
			\parsepgm{s'}             & \text{als $s=c~s'$ met $c$ whitespace}\\
			\parsestr{s'}             & \text{als $s=\texttt{"}~s'$} \\
			\StmCat:\parsepgm{s'}     & \text{als $s=\texttt{+}~s'$} \\
			\StmHead:\parsepgm{s'}    & \text{als $s=\texttt{h}~s'$} \\
			\StmTail:\parsepgm{s'}    & \text{als $s=\texttt{t}~s'$} \\
			\StmQuotify:\parsepgm{s'} & \text{als $s=\texttt{q}~s'$} \\
			\StmPut:\parsepgm{s'}     & \text{als $s=\texttt{p}~s'$} \\
			\StmGet:\parsepgm{s'}     & \text{als $s=\texttt{g}~s'$} \\
			\StmInput:\parsepgm{s'}   & \text{als $s=\texttt{i}~s'$} \\
			\StmOutput:\parsepgm{s'}  & \text{als $s=\texttt{o}~s'$} \\
			\StmExec:\parsepgm{s'}    & \text{als $s=\texttt{x}~s'$} \\
		\end{cases}
$$

Het tweede geval van $\parsepgmop$ zorgt ervoor dat een programma-string
bijvoorbeeld spaties mag bevatten, die syntactisch zelf geen betekenis hebben.
Dit is in overeenkomst met de specificatie, maar op zich niet nodig.

$$
	\parsestr s =
		\begin{cases}
			\lambda:\parsepgm{s'} & \text{als $s=\texttt{"}~s'$} \\
			\unescape{c}~\parsestr{s'} & \text{als $s=\texttt{\textbackslash}~c~s'$
				met $c \in\Char$} \\
			c~\parsestr{s'} & \text{als $s=c~s'$ met $c
				\in\Char\setminus\{\texttt{"},\texttt{\textbackslash}\}$}\\
		\end{cases}
$$

Het tweede geval van $\parsestrop$ zorgt ervoor dat ge-escapete
aanhalingstekens de string niet beëindigen.

Hierbij gebruiken we $\unescapeop$ om bepaalde karakters te unescapen:

$$
	\unescape c =
		\begin{cases}
			\text{het LF-karakter}      & \text{als $c=\texttt{n}$} \\
			\texttt{"}                  & \text{als $c=\texttt{"}$} \\
			\texttt{\textbackslash}     & \text{als $c=\texttt{\textbackslash}$} \\
			\texttt{\textbackslash~$c$} & \text{anderszins}
		\end{cases}
$$

Het laatste alternatief geeft aan dat `ongeldige escape sequences' worden
behandeld alsof de backslash er twee keer stond. Dit is in overeenstemming met
het commentaar op de specificatie en met de Perl interpreter: %todo referentie
\begin{quote}
	This [the specification] does not specify the behaviour of invalid escape
	sequences. The Perl interpreter treats invalid escape sequences as if the
	backslash had occured twice - that is, \textbackslash X is treated as
	\textbackslash\textbackslash X. For maximum compatibility, Smurf programs
	should not rely on this behaviour and should always ensure valid escape
	sequences are used.
\end{quote}

