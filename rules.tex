% vim: set spelllang=nl:
\section{Regels} %todo working title
\label{sec:rules}

We zullen nu ieder syntaxelement nader specificeren. Ook zullen regels voor de
natuurlijke semantiek van Smurf worden geïntroduceerd.

De documentatie \cite{safalra} beschrijft niet wat er gebeurt wanneer er niet
genoeg argumenten op de stack staan om een bepaalde instructie uit te voeren.
We kiezen ervoor om het in zulk soort gevallen onmogelijk te maken een
afleidingsboom te maken (in tegenstelling tot bijvoorbeeld een errorstatus aan
de rechterkant van transities toe te voegen), omdat dit het redeneren over
Smurfprogramma's makkelijker zal maken.

%% vim: set spelllang=nl:
\subsection{\texttt{Push}}
\label{sec:rules:push}

\begin{quote}
	"text" - Places the string on top of the stack (without the quotes). The
	string may include the following escape sequences:
	
	\verb$\n$ = newline \\
	\verb$\"$ = the \verb$"$ character \\
	\verb$\\$ = the \verb$\$ character.
\end{quote}

De string tussen de aanhalingstekens word op de stack gezet, nadat escape
sequences eruit zijn gehaald middels de hulpfunctie $\unescapeop$. Het is,
zoals het commentaar op de specificatie \cite{safalra} aangeeft, niet
gedefinieerd wat er met ongeldige escape sequences gebeurt.

Dit geeft de volgende regel:

$$
\begin{prooftree}
	\trans
		{\pgm}{\ip}{(\push{s}{\stk}), \str)}
		{\ip'}{\op}{\st}
	\justifies
	\trans
		{\StmPush~s:\pgm}{\ip}{(\stk,\str)}
		{\ip'}{\op}{\st}
	\using{\rpushns}
\end{prooftree}
$$

De definitie van $\unescapeop$ is als volgt:

$$
	\unescape c =
		\begin{cases}
			\text{het LF-karakter}      & \text{als $c=\texttt{n}$} \\
			\texttt{"}                  & \text{als $c=\texttt{"}$} \\
			\texttt{\textbackslash}     & \text{als $c=\texttt{\textbackslash}$} \\
			\texttt{\textbackslash~$c$} & \text{anderszins}
		\end{cases}
$$

Het laatste alternatief geeft aan dat `ongeldige escape sequences' worden
behandeld alsof de backslash er twee keer stond. Dit is in overeenstemming met
het commentaar op de specificatie en met de Perl interpreter: %todo referentie
\begin{quote}
	This [the specification] does not specify the behaviour of invalid escape
	sequences. The Perl interpreter treats invalid escape sequences as if the
	backslash had occured twice - that is, \textbackslash X is treated as
	\textbackslash\textbackslash X. For maximum compatibility, Smurf programs
	should not rely on this behaviour and should always ensure valid escape
	sequences are used.
\end{quote}
 

%% vim: set spelllang=nl:
\subsection{\texttt{Head}}
\label{sec:rules:head}

\begin{quote}
	h - Pops a string from the stack, and pushes its head, ie [sic] the first
	character. This causes an error if used on the empty string.
\end{quote}

In plaats van het geven van een error kiezen we ervoor te voorkomen dat we een
afleidingsboom kunnen maken wanneer $\StmHead$ wordt uitgevoerd op het moment
dat het element bovenop de stack de lege string is.

Dit geeft de volgende regel:
\therheadns%

%% vim: set spelllang=nl:
\subsection{\texttt{Tail}}
\label{sec:rules:tail}

\begin{quote}
    t - Pops a string from the stack, and pushes its tail, ie all but the first character. This causes an error if used on the empty string.
\end{quote}

In plaats van het geven van een error kiezen we ervoor te voorkomen dat we een
afleidingsboom kunnen maken wanneer $\StmTail$ wordt uitgevoerd op het moment
dat het element bovenop de stack de lege string is.

Dit geeft de volgende regel:
\thertailns%

%% vim: set spelllang=nl:
\subsection{\texttt{Quotify}}
\label{sec:rules:quotify}

\begin{quote}
	q - "Quotifies" the string on top of the stack, so that it can be placed into
	a Smurf program as a literal string, eg \texttt{Arthur "two-sheds" Jackson}
	becomes \texttt{"Arthur $\backslash$"two-sheds$\backslash$" Jackson"}.
\end{quote}


Er worden aanhalingstekens om de string bovenop de stack gezet. Als er in de
oorspronkelijke string aanhalingstekens, backslashes of LF-karakters staan, dan
wordt hier een \verb$\$ voor geplaatst. Hiervoor gebruiken we de hulpfunctie
$\escapeop$.

Dit geeft de volgende regel:
$$
\begin{prooftree}
	\trans
		{\pgm}{\ip}{(\push{\texttt{"}\escape{s}\texttt{"}}{\stk'}, \str) }
		{\ip'}{\op}{\st}
	\justifies
	\trans
		{\StmQuotify:\pgm}{\ip}{(\stk,\str)}
		{\ip'}{\op}{\st}
	\using{\rquotifyns}
	\qquad
	\text{met $\pop{\stk} = (s, stk')$.}
\end{prooftree}
$$

$$
	\escape{c~s} =
		\begin{cases}
			\texttt{\textbackslash}~c~\escape{s} & \text{als
				$c\in\{\texttt{"},\texttt{\textbackslash},\text{het LF-karakter}\}$} \\
			c~\escape{s}                & \text{anderszins}
		\end{cases}
$$

%% vim: set spelllang=nl:
\subsection{\texttt{Cat}}
\label{sec:rules:cat}

\begin{quote}
    + - concatenates the top two strings on the stack. The string pushed earlier
		appears earlier in the resulting string, eg \smurfinline{"Zork" "mid" +}
		would result in the string \texttt{Zorkmid} being placed on the stack.
\end{quote}

De string bovenop de stack wordt toegevoegd aan de string hieronder. Het
resultaat wordt op de stack gezet.

Dit geeft de volgende regel:

$$
\begin{prooftree}
	\trans
		{\pgm}{\ip}{(\push{s1~s2}{\stk''}, \str)}
		{\ip'}{\op}{\st}
	\justifies
	\trans
		{\StmCat:\pgm}{\ip}{(\stk,\str)}
		{\ip'}{\op}{\st}
	\using{\rcatns}
	\qquad
	\text{met\enspace
	\parbox{36mm}{$\pop{\stk} = (s_2,\stk') $,\\$ \pop{\stk'} = (s_1,\stk'')$.}}
\end{prooftree}
$$


% vim: set spelllang=nl:
\subsection{\texttt{Get}}
\label{sec:rules:exec}

\begin{quote}
	g - Pops a variable name from the stack, and pushes the value of the
	variable.
\end{quote}

De regel voor dit statement spreekt voor zich. In het geval dat $\stk$ leeg is,
is $\pop\stk$ niet gedefinieerd en kunnen we de regel dus niet toepassen. Omdat
$\StmGet$ geen IO gebruikt kunnen we $\ip$, $\ip'$ en $\op$ direct doorgeven.

$$
\begin{prooftree}
	\trans
		{\pgm}{\ip}{(\push{\str~\var}{\stk'}, \str)}
		{\ip'}{\op}{\st}
	\justifies
	\trans
		{\StmGet:\pgm}{\ip}{(\stk,\str)}
		{\ip'}{\op}{\st}
	\using{\rgetns}
	\qquad
	\text{met $(\var,\stk') = \pop{\stk}$.}
\end{prooftree}
$$


% vim: set spelllang=nl:
\subsection{\texttt{Put}}
\label{sec:rules:put}

\begin{quote}
	p - Pops a variable name from the stack, pops a value from the stack, and
	assigns that value to the variable name.
\end{quote}

We halen twee strings van de stack en gebruiken $\putop$ om een nieuwe variable
store te krijgen. Hiermee wordt de rest van het programma uitgevoerd. Als er
minder dan twee elementen op de stack staan kan deze regel niet worden
toegepast, aangezien $\popop$ een partiële functie is.

$$
\begin{prooftree}
	\trans
		{\pgm}{\ip}{(\stk'', \put\var\val\str)}
		{\ip'}{\op}{\st}
	\justifies
	\trans
		{\StmPut:\pgm}{\ip}{(\stk,\str)}
		{\ip'}{\op}{\st}
	\using{\rputns}
	\qquad
	\text{met\enspace
		\parbox{36mm}{$\pop{\stk} = (\var,\stk')$,
		\\$\pop{\stk'}= (\val,\stk'')$.}}
\end{prooftree}
$$


%% vim: set spelllang=nl:
\subsection{\texttt{Input}}
\label{sec:rules:input}

\begin{quote}
	i - takes a string from user input, and places it on the stack.
\end{quote}

Dit commando is erg slecht gedefinieerd. Mag deze input
bijvoorbeeld line feeds bevatten? Een naïeve implementatie die van stdin
gebruik maakt zou standaard splitsen op line feeds en dit dus niet toestaan.
Een andere implementatie zou gebruik kunnen maken van grafische prompts waarin
het wél mogelijk is meerdere regels input te geven.

Om bij het redeneren over de semantiek geen last te hebben van zulk soort
implementatiedetails hebben we ervoor gekozen hier tot op een hoog niveau van
te abstraheren. We gaan ervan uit dat we een stack van strings, $\Input$,
hebben die ons de input geeft. Hoe deze stack is geïmplementeerd doet niet ter
zake. Ook leggen we geen restricties op aan de karakters waaruit haar elementen
bestaan.

We gaan er verder van uit dat de gebruiker zijn programma volledig wil
uitvoeren en dus voldoende input zal geven, waardoor $\pop\ip$ altijd
gedefinieerd is. Geeft de gebruiker niet genoeg input, dan hoeven we dus geen
afleidingsboom te maken (en we zullen dit ook onmogelijk maken).

Dit geeft de volgende regel:

$$
\begin{prooftree}
	\trans
		{\pgm}{\ip'}{(\push\val\stk, \str)}
		{\ip''}{\op}{\st}
	\justifies
	\trans
		{\StmInput:\pgm}{\ip}{(\stk,\str)}
		{\ip''}{\op}{\st}
	\using{\rinputns}
	\qquad
	\text{met $\pop\ip = (\val,\ip')$.}
\end{prooftree}
$$


%% vim: set spelllang=nl:
\subsection{\texttt{Output}}

\begin{quote}
	o - Output the string at the top of the stack
\end{quote}

Net als bij het inputcommando gaan we op een abstracte wijze met de output om.
We houden gedurende het hele programma een stack van strings, genaamd $\Output$
bij waar het programma zijn output naar wegschrijft.

Dit geeft de volgende regel:
\theroutputns%

waarbij $\op$ in de bovenste regel de gehele outputstack weergeeft. Merk op
dat eenzelfde regel waar $s$ niet voor op de stack wordt gezet maar achter,
even geldig is. Geen van beide opties is beter dan de ander omdat we geen
aannames doen over hoe de $\Output$-stack wordt verwerkt.

% vim: set spelllang=nl:
\subsection{\texttt{Exec}}
\label{sec:rules:exec}

\begin{quote}
	x - Executes the string at the top of the stack as a Smurf program. The stack
	and variable store are erased.
\end{quote}

We halen een string van de stack en gebruiken $\parsepgmop$ om dit in een
programma om te zetten. Hieronder zullen we $\parsepgmop$ definiëren. Dit wordt
het nieuwe programma om uitgevoerd te worden.  Als de $\stk$ leeg is is deze
regel niet toepasbaar, omdat $\pop\stk$ dan niet gedefinieerd is. Ook is deze
regel niet toepasbaar als de gepopte string zelf geen geldig Smurf-programma
is, omdat $\parsepgmop$ dan niet gedefinieerd is.

$$
\begin{prooftree}
	\trans
		{\pgm'}{\ip}{(\Nil, \emptystore)}
		{\ip'}{\op}{\st}
	\justifies
	\trans
		{\StmExec:\pgm}{\ip}{(\stk,\str)}
		{\ip'}{\op}{\st}
	\using{\rexecns}
	\qquad
	\text{met\enspace
		\parbox{36mm}{$ \pop{\stk} =(\var,\stk')$,\\
			$\pgm' = \parsepgm{\var'}$.}
	}
\end{prooftree}
$$

\medskip
$\parsepgmop$ definiëren we als volgt, met een hulpfunctie $\parsestrop$:

$$
	\parsepgm s =
		\begin{cases}
			\lambda                   & \text{als $s=\lambda$}\\
			\parsepgm{s'}             & \text{als $s=c~s'$ met $c$ whitespace}\\
			\parsestr{s'}             & \text{als $s=\texttt{"}~s'$} \\
			\StmCat:\parsepgm{s'}     & \text{als $s=\texttt{+}~s'$} \\
			\StmHead:\parsepgm{s'}    & \text{als $s=\texttt{h}~s'$} \\
			\StmTail:\parsepgm{s'}    & \text{als $s=\texttt{t}~s'$} \\
			\StmQuotify:\parsepgm{s'} & \text{als $s=\texttt{q}~s'$} \\
			\StmPut:\parsepgm{s'}     & \text{als $s=\texttt{p}~s'$} \\
			\StmGet:\parsepgm{s'}     & \text{als $s=\texttt{g}~s'$} \\
			\StmInput:\parsepgm{s'}   & \text{als $s=\texttt{i}~s'$} \\
			\StmOutput:\parsepgm{s'}  & \text{als $s=\texttt{o}~s'$} \\
			\StmExec:\parsepgm{s'}    & \text{als $s=\texttt{x}~s'$} \\
		\end{cases}
$$

Het tweede geval van $\parsepgmop$ zorgt ervoor dat een programma-string
bijvoorbeeld spaties mag bevatten, die syntactisch zelf geen betekenis hebben.
Dit is in overeenkomst met de specificatie, maar op zich niet nodig.

$$
	\parsestr s =
		\begin{cases}
			\lambda:\parsepgm{s'} & \text{als $s=\texttt{"}~s'$} \\
			\unescape{c}~\parsestr{s'} & \text{als $s=\texttt{\textbackslash}~c~s'$
				met $c \in\Char$} \\
			c~\parsestr{s'} & \text{als $s=c~s'$ met $c
				\in\Char\setminus\{\texttt{"},\texttt{\textbackslash}\}$}\\
		\end{cases}
$$

Het tweede geval van $\parsestrop$ zorgt ervoor dat ge-escapete
aanhalingstekens de string niet beëindigen. Hierbij gebruiken we $\unescapeop$
om bepaalde karakters te unescapen. Zie voor de definitie van deze regel
\autoref{sec:rules:push}.



