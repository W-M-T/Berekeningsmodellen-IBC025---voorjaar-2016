% vim: set spelllang=nl:
\section{Analyse}
\label{sec:analyse}
Als analyse willen we graag een stuk code dat een string omdraait bekijken aan
de hand van onze semantiekregels. Deze code ziet er als volgt uit:
\begin{smurf}
	\footnotesize
	"+"i+ ""p ""gtg ""gt "i"p\\
	"\textbackslash{}"\textbackslash{}"p\textbackslash{}"i\textbackslash{}"gh\textbackslash{}"o\textbackslash{}"g+\textbackslash{}"o\textbackslash{}"p\textbackslash{}"i\textbackslash{}"gt\textbackslash{}"i\textbackslash{}"p\textbackslash{}"\textbackslash{}\textbackslash{}\textbackslash{}"+\textbackslash{}\textbackslash{}\textbackslash{}"\textbackslash{}\textbackslash{}\textbackslash{}\textbackslash{}\textbackslash{}\textbackslash{}"\textbackslash{}\textbackslash{}\textbackslash{}"p\textbackslash{}"\textbackslash{}"i\textbackslash{}"gq+\textbackslash{}"tg\textbackslash{}"+\textbackslash{}"i\textbackslash{}"gq+\textbackslash{}"\textbackslash{}\textbackslash{}\\
    \textbackslash{}"i\textbackslash{}\textbackslash{}\textbackslash{}"p\textbackslash{}"+\textbackslash{}"o\textbackslash{}"gq+\textbackslash{}"\textbackslash{}\textbackslash{}\textbackslash{}"o\textbackslash{}\textbackslash{}\textbackslash{}"p\textbackslash{}"+\textbackslash{}"\textbackslash{}"gq+\textbackslash{}"\textbackslash{}"g+\textbackslash{}"\textbackslash{}"p\textbackslash{}"o\textbackslash{}"gq\textbackslash{}"o\textbackslash{}"+\textbackslash{}"+\textbackslash{}"pgx"\\
	""p "\textbackslash{}"+\textbackslash{}"\textbackslash{}"\textbackslash{}"p" "i"gq+ "tg"+ "i"gq+\\
	"\textbackslash{}"i\textbackslash{}"p\textbackslash{}"\textbackslash{}""+ ""gq+ ""g+ ""p "" "+" "i"g+ pgx
\end{smurf}

Omgezet naar onze leesbare (hier een relatief begrip) syntax ziet de code er zo uit:

\begin{smurf}
	\footnotesize
	
	\StmPush~"+"~: 
	\StmInput~: 
	\StmCat~: 
	\StmPush~$\lambda$~: 
	\StmPut~: 
	\StmPush~$\lambda$~: 
	\StmGet~: 
	\StmTail~:\\
	\StmGet~: 
	\StmPush~$\lambda$~: 
	\StmGet~: 
	\StmTail~: 
	\StmPush~"i"~: 
	\StmPut~:\\
	\StmPush~"\textbackslash{}"\textbackslash{}"p\textbackslash{}"i\textbackslash{}"gh\textbackslash{}"o\textbackslash{}"g+\textbackslash{}"o\textbackslash{}"p\textbackslash{}"i\textbackslash{}"gt\textbackslash{}"i\textbackslash{}"p\textbackslash{}"\textbackslash{}\textbackslash{}\textbackslash{}"+\textbackslash{}\textbackslash{}\textbackslash{}"\textbackslash{}\textbackslash{}\textbackslash{}"\textbackslash{}\textbackslash{}\textbackslash{}"p\textbackslash{}"\textbackslash{}"i\textbackslash{}"gq+\textbackslash{}"tg\textbackslash{}"+\textbackslash{}"i\textbackslash{}"gq+\textbackslash{}"\textbackslash{}\textbackslash{}\\
    \textbackslash{}"i\textbackslash{}\textbackslash{}\textbackslash{}"p\textbackslash{}"+\textbackslash{}"o\textbackslash{}"gq+\textbackslash{}"\textbackslash{}\textbackslash{}\textbackslash{}"o\textbackslash{}\textbackslash{}\textbackslash{}"p\textbackslash{}"+\textbackslash{}"\textbackslash{}"gq+\textbackslash{}"\textbackslash{}"g+\textbackslash{}"\textbackslash{}"p\textbackslash{}"o\textbackslash{}"gq\textbackslash{}"o\textbackslash{}"+\textbackslash{}"+\textbackslash{}"pgx"~:\\
	\StmPush~$\lambda$~: 
	\StmPut~: 
	\StmPush~"\textbackslash{}"+\textbackslash{}"\textbackslash{}"\textbackslash{}"p"~:\\
	\StmPush~"i"~: 
	\StmGet~: 
	\StmQuotify~: 
	\StmCat~: 
	\StmPush~"tg"~: 
	\StmCat~: 
	\StmPush~"i"~: 
	\StmGet~: 
	\StmQuotify~: 
	\StmCat~: 
	\StmPush~"\textbackslash{}"i\textbackslash{}"p\textbackslash{}"\textbackslash{}""~: 
	\StmCat~: 
	\StmPush~$\lambda$~: 
	\StmGet~: 
	\StmQuotify~: 
	\StmCat~: 
	\StmPush~$\lambda$~: 
	\StmGet~: 
	\StmCat~: 
	\StmPush~$\lambda$~: 
	\StmPut~: 
	\StmPush~$\lambda$~: 
	\StmPush~"+"~: 
	\StmPush~"i"~: 
	\StmGet~: 
	\StmCat~: 
	\StmPut~: 
	\StmGet~: 
	\StmExec~: 
    $\lambda$

    
\end{smurf}

Nu laten we zien dat deze code daadwerkelijk alle mogelijke strings omdraait, oftewel:\\
Er is een derivatieboom voor
$$
\trans
	{Programma}{[s:\Nil]}{(\Nil,\emptystore)}
	{\Nil}{s^R}{\st}
$$
voor alle s, waar
$$(c~s)^R=s^R c$$
$$\lambda^R=\lambda$$

Dit bewijzen we door middel van inductie op de lengte van s.

Basisgeval: s = $\lambda$
$$%Todo fix dit
\begin{prooftree}
	\[
    	\[
        	\[
            	\[
                	test
                	\justifies
                    \trans
        				{\StmPut : \StmPush~\lambda : ...~}{\Nil}{([\lambda:["+":\Nil]], \emptystore)}
        				{\Nil}{\lambda}{\st}
                    \using{\rputns}
                \]
            	\justifies
        		\trans
        			{\StmPush~\lambda : \StmPut : ...~}{\Nil}{(["+":\Nil], \emptystore)}
        			{\Nil}{\lambda}{\st}
                \using{\rpushns}
            \]
        	\justifies
    		\trans
        		{\StmCat : \StmPush~\lambda : ...~}{\Nil}{([\lambda:["+":\Nil]], \emptystore)}
        		{\Nil}{\lambda}{\st}
            \using{\rcatns}
        \]
    	\justifies
        \trans
        	{\StmInput : \StmCat : ...~}{[\lambda:\Nil]}{(["+":\Nil], \emptystore)}
        	{\Nil}{\lambda}{\st}
        \using{\rinputns}
    \]
	\justifies
	\trans
		{\StmPush~"+" : \StmInput : ...~}{[\lambda:\Nil]}{(\Nil,\emptystore)}
		{\Nil}{\lambda}{\st}
	\using{\rpushns}
\end{prooftree}
$$

	
We willen hierbij een stuk code onderzoeken dat volgens de specificaties geen
eenduidige uitkomst heeft. Het liefst willen we eens stuk code bedenken waarbij
onder verschillende interpretaties van de specificaties een geheel andere
uitkomst mogelijk is. Wellicht kunnen we daarna nieuwe specificaties (gebaseerd
op de bestaande specificaties) maken die wel eenduidig zijn en aansluiten op
onze semantiekregels.
