% vim: set spelllang=nl:
\section{Analyse}
\label{sec:analyse}

Omdat de transities van onze natuurlijke semantiek input meenemen, kunnen we
alleen een afleidingsboom maken voor een programma \emph{met} een bepaalde
input. Het is dus niet triviaal mogelijk een afleidingsboom te maken voor
willekeurige input. In deze sectie willen we laten zien hoe het toch mogelijk
is een bewijs te leveren over een programma met een willekeurige input string,
door gebruik te maken van inductie naar de lengte van de input string. We
hadden hiervoor initieel een programma op de Esolang wiki gevonden dat een
string zou omdraaien~\cite{esolang:prog}. Dit programma werkte echter niet naar
behoren: het werkte niet voor strings met lengte $1$. Daarom hebben we zelf een
programma geschreven voor het omdraaien van een string. Dit programma ziet er
als volgt uit:

\begin{smurf}
\footnotesize
i "input" p
"input" g
"input" g q
"\textbackslash{}"\textbackslash{}"\textbackslash{}"\textbackslash{}\textbackslash{}\textbackslash{}"u\textbackslash{}\textbackslash{}\textbackslash{}"p\textbackslash{}\textbackslash{}\textbackslash{}"v\textbackslash{}\textbackslash{}\textbackslash{}"p\textbackslash{}\textbackslash{}\textbackslash{}"w\textbackslash{}\textbackslash{}\textbackslash{}"p\textbackslash{}\textbackslash{}\textbackslash{}"w\textbackslash{}\textbackslash{}\textbackslash{}"gh\textbackslash{}\textbackslash{}\textbackslash{}"v\textbackslash{}\textbackslash{}\textbackslash{}"g+\textbackslash{}\textbackslash{}\textbackslash{}"v\textbackslash{}\textbackslash{}\textbackslash{}"p\textbackslash{}\textbackslash{}\textbackslash{}"w\textbackslash{}\textbackslash{}\textbackslash{}"gt 

\textbackslash{}\textbackslash{}\textbackslash{}"w\textbackslash{}\textbackslash{}\textbackslash{}"p\textbackslash{}\textbackslash{}\textbackslash{}"w\textbackslash{}\textbackslash{}\textbackslash{}"g\textbackslash{}\textbackslash{}\textbackslash{}"v\textbackslash{}\textbackslash{}\textbackslash{}"gq\textbackslash{}\textbackslash{}\textbackslash{}"o\textbackslash{}\textbackslash{}\textbackslash{}"+\textbackslash{}\textbackslash{}\textbackslash{}"w\textbackslash{}\textbackslash{}\textbackslash{}"gq\textbackslash{}\textbackslash{}\textbackslash{}"v\textbackslash{}\textbackslash{}\textbackslash{}"gq\textbackslash{}\textbackslash{}\textbackslash{}"u\textbackslash{}\textbackslash{}\textbackslash{}"gq++\textbackslash{}\textbackslash{}\textbackslash{}"u\textbackslash{}\textbackslash{}\textbackslash{}"g+

\textbackslash{}\textbackslash{}\textbackslash{}"w\textbackslash{}\textbackslash{}\textbackslash{}"gp\textbackslash{}\textbackslash{}\textbackslash{}"\textbackslash{}\textbackslash{}\textbackslash{}"pgx\textbackslash{}"\textbackslash{}"u\textbackslash{}"p\textbackslash{}"v\textbackslash{}"p\textbackslash{}"w\textbackslash{}"p\textbackslash{}"w\textbackslash{}"gh\textbackslash{}"v\textbackslash{}"g+\textbackslash{}"v\textbackslash{}"p\textbackslash{}"w\textbackslash{}"gt\textbackslash{}"w\textbackslash{}"p\textbackslash{}"w\textbackslash{}"g\textbackslash{}"v\textbackslash{}"gq\textbackslash{}"o\textbackslash{}"+

\textbackslash{}"w\textbackslash{}"gq\textbackslash{}"v\textbackslash{}"gq\textbackslash{}"u\textbackslash{}"gq++\textbackslash{}"u\textbackslash{}"g+\textbackslash{}"w\textbackslash{}"gp\textbackslash{}"\textbackslash{}"pgx"
\\
+
"input" g p
"\textbackslash{}"\textbackslash{}"o" "" p
g x
\end{smurf}

Het bovenstaande programma is correcte Smurfsyntax en hierdoor niet erg
leesbaar. We hebben daarom getracht het programma iets leesbaarder te maken: 

\begin{smurf}
\footnotesize
\StmInput~: \StmPush~"input"~: \StmPut~: \StmPush~"input"~: \StmGet~:
\StmPush~"input"~: \StmGet~: \StmQuotify~:\\ \StmPush~
"\textbackslash{}"\textbackslash{}"\textbackslash{}"\textbackslash{}\textbackslash{}\textbackslash{}"u\textbackslash{}\textbackslash{}\textbackslash{}"p\textbackslash{}\textbackslash{}\textbackslash{}"v\textbackslash{}\textbackslash{}\textbackslash{}"p\textbackslash{}\textbackslash{}\textbackslash{}"w\textbackslash{}\textbackslash{}\textbackslash{}"p\textbackslash{}\textbackslash{}\textbackslash{}"w\textbackslash{}\textbackslash{}\textbackslash{}"gh\textbackslash{}\textbackslash{}\textbackslash{}"v\textbackslash{}\textbackslash{}\textbackslash{}"g+\textbackslash{}\textbackslash{}\textbackslash{}"v\textbackslash{}\textbackslash{}\textbackslash{}"p\textbackslash{}\textbackslash{}\textbackslash{}"w\textbackslash{}\textbackslash{}\textbackslash{}"gt

\textbackslash{}\textbackslash{}\textbackslash{}"w\textbackslash{}\textbackslash{}\textbackslash{}"p\textbackslash{}\textbackslash{}\textbackslash{}"w\textbackslash{}\textbackslash{}\textbackslash{}"g\textbackslash{}\textbackslash{}\textbackslash{}"v\textbackslash{}\textbackslash{}\textbackslash{}"gq\textbackslash{}\textbackslash{}\textbackslash{}"o\textbackslash{}\textbackslash{}\textbackslash{}"+\textbackslash{}\textbackslash{}\textbackslash{}"w\textbackslash{}\textbackslash{}\textbackslash{}"gq\textbackslash{}\textbackslash{}\textbackslash{}"v\textbackslash{}\textbackslash{}\textbackslash{}"gq\textbackslash{}\textbackslash{}\textbackslash{}"u\textbackslash{}\textbackslash{}\textbackslash{}"gq++\textbackslash{}\textbackslash{}\textbackslash{}"u\textbackslash{}\textbackslash{}\textbackslash{}"g+

\textbackslash{}\textbackslash{}\textbackslash{}"w\textbackslash{}\textbackslash{}\textbackslash{}"gp\textbackslash{}\textbackslash{}\textbackslash{}"\textbackslash{}\textbackslash{}\textbackslash{}"pgx\textbackslash{}"\textbackslash{}"u\textbackslash{}"p\textbackslash{}"v\textbackslash{}"p\textbackslash{}"w\textbackslash{}"p\textbackslash{}"w\textbackslash{}"gh\textbackslash{}"v\textbackslash{}"g+\textbackslash{}"v\textbackslash{}"p\textbackslash{}"w\textbackslash{}"gt\textbackslash{}"w\textbackslash{}"p\textbackslash{}"w\textbackslash{}"g\textbackslash{}"v\textbackslash{}"gq\textbackslash{}"o\textbackslash{}"+

\textbackslash{}"w\textbackslash{}"gq\textbackslash{}"v\textbackslash{}"gq\textbackslash{}"u\textbackslash{}"gq++\textbackslash{}"u\textbackslash{}"g+\textbackslash{}"w\textbackslash{}"gp\textbackslash{}"\textbackslash{}"pgx"~:
\\
\StmCat~: \StmPush~"input" \StmGet~: \StmPut
: \StmPush~"\textbackslash{}"\textbackslash{}"o"~: \StmPush~""~: \StmPut~:
\StmGet~: \StmExec~: $\lambda$

\end{smurf}

% vim: set spelllang=nl:
\subsection{uitleg programma}
\label{sec:uitleg programma}

\subsection{bewijs} 
Nu laten we zien dat deze code daadwerkelijk alle mogelijke strings omdraait,
oftewel: er is een afleidingsboom voor
$$
\trans
	{Programma}{[s:\Nil]}{(\Nil,\emptystore)}
	{\Nil}{[s^R:\Nil]}{\st}
$$

voor alle s, waar
$$(c~s)^R=s^R c$$
$$\lambda^R=\lambda$$


Dit bewijs gaat in twee stappen.

$$
\trans
	{Programma}{[c:s:\Nil]}{(\Nil,\emptystore)}
	{\Nil}{[s^R:c:\Nil]}{\st}
$$



Eerst bewijzen we aan de hand van inductie dat voor het recursieve binnenste programma geldt dat voor de stack met waarde $(s:t:\Nil)$ de output $(t^R: s^R: \Nil)$  zal zijn. 

\bigskip 

Ook wel
$$
\trans
	{Programma}{[cs:g:\Nil]}{(\Nil,\emptystore)}
	{\Nil}{[g^R:cs:\Nil]}{\st}
$$

\begin{proof}[Bewijs]
\medskip
	Met inductie naar de lengte van $g$.

	Basisgeval: $g=\lambda$.
	
	We moeten laten zien dat 
	
	$$
    \trans
    	{Programma}{[s:c:\Nil]}{(\Nil,\emptystore)}
    	{\Nil}{[cs:\Nil]}{\st}
    $$
    
    Dit volgt uit bewijsboom *referentie*
    \bigskip
    
    inductiestap: $g = t $ (t is een variabele)
    
    We moeten laten zien dat
    $$
    \trans
	{Programma}{[s:ct:\Nil]}{(\Nil,\emptystore)}
	{\Nil}{[t^R:cs:\Nil]}{\st}
    $$

    Met als inductiehypothese 
     $$
    \trans
	{Programma}{[cs:t:\Nil]}{(\Nil,\emptystore)}
	{\Nil}{[t^R:cs:\Nil]}{\st}
    $$
    
    Dit volgt uit bewijsboom * referentie*
	
\end{proof}

Vervolgens zullen we laten zien dat voor het buitenste programma geldt dat 

$$
\trans
	{Programma}{[s:\Nil]}{(\Nil,\emptystore)}
	{\Nil}{[s^R:\Nil]}{\st}
$$

Hierbij zullen we laten zien dat dit geldt voor zowel een lege input als een niet-lege input. 
Hierbij mogen we natuurlijk gebruik maken van wat we hierboven bewezen hebben. 
\medskip

Eerst zullen we laten zien dat het geldt voor een lege input 

We zullen dus moeten laten zien dat:
$$
\trans
	{Programma}{[\Nil:\Nil]}{(\Nil,\emptystore)}
	{\Nil}{[\Nil:\Nil]}{\st}
$$

Dit volgt direct uit bewijsboom *referentie* 


Ten tweede zullen we laten zien dat het geldt voor een niet-lege input 
We zullen dus moeten laten zien dat:
$$
\trans
	{Programma}{[s:\Nil]}{(\Nil,\emptystore)}
	{\Nil}{[s^t:\Nil]}{\st}
$$

Dit volgt uit de bewijsbomen *referentie* (3*) 





