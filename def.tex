% vim: set spelllang=nl:
\section{Definities}
\label{sec:def}

De voorbeelden hierboven zijn niet erg leesbaar, wat het lastig maakt om over
Smurfprogramma's te redeneren. We defini\"eren daarom een alternatieve syntax,
die we in de rest van het werkstuk zullen gebruiken. Dit doen we in
\autoref{sec:def:syn}. Vervolgens defini\"eren we hoe we input en output
modelleren (\ref{sec:def:io}), hoe de toestand van een Smurfprogramma eruit
ziet (\ref{sec:def:state}) en wat voor transities we zullen gebruiken bij het
specificeren van Smurf in de natuurlijke semantiek (\ref{sec:def:trans}). Bij
al definities gebruiken we een aantal metavariabelen die worden beschreven in
\autoref{sec:def:meta}.

% vim: set spelllang=nl:
\subsection{Metavariabelen}
\label{sec:def:meta}
We zullen de volgende metavariabelen gebruiken:

\begin{description}[labelindent=\parindent]
	\item[$a$] voor typen,
	\item[$c$] voor karakters ($\Char$),
	\item[$s$] voor strings ($\String$),
	\item[$\var$] voor strings die als naam van een variabele worden gebruikt,
	\item[$\val$] voor strings die als waarde van een variabele worden gebruikt,
	\item[$e$] voor elementen van stacks ($a$ voor een element van $\Stack{a}$),
	\item[$\stk$] voor stacks ($\Stack{a}$ voor willekeurige $a$),
	\item[$\ip$] voor inputstacks ($\Input$),
	\item[$\op$] voor outputstacks ($\Output$),
	\item[$\pgm$] voor programma's ($\Pgm$),
	\item[$\st$] voor toestanden ($\State$) en
	\item[$\str$] voor variable stores.
\end{description}

% vim: set spelllang=nl:
\subsection{Syntax}
\label{sec:def:syn}
We definiëren de volgende syntax:
\setlength{\grammarindent}{5em}
\begin{grammar}
	<Pgm> ::= <Stm>:<Pgm> | $\lambda$

	<Stm> ::= `Push' <String>
		\alt `Cat' | `Head' | `Tail' | `Quotify'
		\alt `Put' | `Get'
		\alt `Input' | `Output'
		\alt `Exec'

	<String> ::= <Char><String> | $\lambda$
\end{grammar}

Een karakter, $\SynChar$, is een symbool uit de ASCII tabel.

Programma's zijn lijsten van statements. Merk op dat compositie van statements
hier expliciet is door middel van de \lit{:}, waar compositie in de
oorspronkelijke versie van de syntax impliciet was.

% vim: set spelllang=nl:
\subsection{Input en output}
\label{sec:def:io}

Allereerst definiëren we het type $\Stack{a}$, omdat stacks we veel met stacks
doen in onze semantiek regels. Een $\Stack{a}$ (lees: een stack van elementen
van type $a$) is een simpel datatype met de volgende syntax:

\def\inbrackets#1{$\mathrm{[}#1\mathrm{]}$}
\def\bracka{\inbrackets{a}}
\begin{grammar}
	<Stack \bracka> ::= [<a>:<Stack \bracka>] | `Nil'
\end{grammar}

Op een stack zijn twee instructies gedefinieerd:
\begin{gather*}
	\pushop : a \times \Stack{a} \to \Stack{a} \\
	\push{e}{\stk} \isdef [e:\stk] \\[1em]
	\popop : \Stack{a} \hookrightarrow a \times \Stack{a} \\
	\pop{[e:\stk]} \isdef (e,\stk) \\
\end{gather*}

$\popop$ is een partiële functie omdat $\pop\Nil$ niet gebruikt mag worden in
onze semantiekregels. %todo waarom niet?

\medskip
We zullen de input en output beide als $\Stack{\String}$ modelleren. In feite
zal zelfs blijken dat we op $\Input$ de operatie $\pushop$ niet nodig hebben,
en op $\Output$ de operatie $\popop$ niet zullen gebruiken. Informeel
beschouwen we $\Input$ als een `bron' van $\String$s en $\Output$ als een `put'
van $\String$s. Formeel:
\begin{align*}
	\Input &\isdef \Stack{\String} \\ \Output &\isdef \Stack{\String}
\end{align*}


% vim: set spelllang=nl:
\subsection{Toestanden}
\label{sec:def:state}

Een toestand $s\in\State$ van Smurf bevat zowel een stack als een variable
store. Hieruit volgt de voor de hand liggende definitie voor $\State$,
$$\State \isdef \Stack{\String} \times (\String \to \String),$$
waarbij we $s=(\stk,\str)\in\State$ lezen als de toestand $s$ met stack $\stk$
en variable store $\str$.

De variable store is een totale functie. Initieel zijn alle waardes in de store
$\lambda$ in overeenstemming met de documentatie \cite{safalra}:

\begin{quote}
	Any string can be used as a variable name, including the empty string. All
	variable values are initially set to the empty string.
\end{quote}

Om de waarde van een key $k$ uit store $\str$ te halen gebruiken we simpelweg
$\str~k$. Vervolgens definiëren we $\putop:\SynString \times \SynString \times
(\String\to\String) \to (\String\to\String)$ die gegeven een variabelenaam, een
waarde en een oude store een nieuwe store oplevert:

$$
	\put{\var}{\val}{\str} k =
		\begin{cases}
			\val   & \text{als $k=\var$} \\
			\str~k & \text{als $k\ne\var$}
		\end{cases}
$$


% vim: set spelllang=nl:
\subsection{Transities}
\label{sec:def:trans}
We hebben ervoor gekozen de semantiek van Smurf in natuurlijke semantiek te
definiëren. In principe hadden we er ook voor kunnen kiezen om structurele
operationele semantiek te gebruiken. In het geval van Smurf komt dit ongeveer
op hetzelfde neer. Hoe regels voor Smurf gedefinieerd zouden kunnen worden in
structurele operationele semantiek wordt toegelicht in \autoref{sec:sos}.

Bij het definiëren van de natuurlijke semantiek van Smurf zullen we de
verzameling van transities als een relatie $\to$ tussen
$\Pgm\times\Input\times\State$ en $\Input\times\Output\times\State$ beschouwen.
Dit schrijven we als
$$\trans{\pgm}{\ip}{\st}{\ip'}{\op}{\st'}$$
en lezen we als
\begin{quote}
	``het programma $\pgm$ zal in toestand $\st$ gegeven input $\ip$ leiden tot
	toestand $\st'$, met output $\op$ waarbij $\ip'$ gelijk is aan $\ip$ zonder
	de geconsumeerde input.''
\end{quote}

We hebben het hele programma $\pgm$ nodig voor de pijl, omdat één commando
($\StmExec$) eventuele verdere statements `weggooit'. %todo ander woord
Verder gebruiken we $\Input$ voor $\StmInput$ en hebben we natuurlijk de
$\State$ nodig voor ieder statement: ieder statement verandert de stack, en
sommige statements veranderen bovendien de variable store.


