% vim: set spelllang=nl:
\subsection{Organisatie} %todo titel

In \autoref{sec:def} beschrijven we formele definities om de semantiek van
Smurf te kunnen specificeren. We kijken naar de syntax, input en output, de
programmatoestand en de transities in de natuurlijke semantiek die we gaan
definiëren. \autoref{sec:rules} beschrijft vervolgens per statement in de
syntax de formele natuurlijke semantiek. Hierbij baseren we ons op de Smurf
specificatie \cite{safalra}, waarbij we dingen verhelderen en ongedefinieerd
gedrag definiëren. In \autoref{sec:sos} laten we zien wat voor regels we zouden
moeten gebruiken als we structurele operationele semantiek zouden gebruiken. In
\autoref{sec:analyse} bekijken we een stuk code aan de hand van de
gedefinieerde regels.

\autoref{sec:planning} bevat een planning voor het afwerken van het werkstuk,
die uiteindelijk verwijderd zal worden.

