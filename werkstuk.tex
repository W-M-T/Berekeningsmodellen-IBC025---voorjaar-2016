% vim: set spelllang=nl:
\documentclass[a4paper]{article}

\title{De semantiek van Smurf} %todo working title
\author{Evi Sijben, Ward Theunisse en Camil Staps}

% Standaard packages
\usepackage[hidelinks]{hyperref}
\usepackage[utf8]{inputenc}
\usepackage[dutch,shorthands=off]{babel}
\usepackage{geometry}

% Taakspecifieke packages
\usepackage{amsmath}
\usepackage{amsthm}
\usepackage{enumitem}
\usepackage{pdflscape}
\usepackage{prooftree}
\usepackage{stackrel}
\usepackage{syntax}
\usepackage{thmtools}

\usepackage{clean}
\lstset{language=Clean,breaklines,tabsize=2,xleftmargin=\parindent}

% Eigen packages
\usepackage{smurf}

% Settings, fixes
\setlist{itemsep=0pt}
\addto\extrasdutch{%
	\renewcommand{\sectionautorefname}{Sectie}
	\renewcommand{\subsectionautorefname}{Paragraaf}
}

% Eigen commando's en environments, niet specifiek voor Smurf
\declaretheoremstyle[
	title=Voorbeeld,
	parent=section,
	spacebelow=1em,
	preheadhook=\nobreak\noindent\hrulefill,
	prefoothook=\vspace*{\dimexpr-\baselineskip+\topsep\relax}\endgraf\nobreak\noindent\hrulefill%
]{lined}
\declaretheorem[style=lined]{exmp}

\begin{document}

\maketitle

% vim: set spelllang=nl:
\begin{abstract}
	We beschrijven een manier om de semantiek van Smurf \cite{safalra} formeel te
	specificeren, om het makkelijker te maken over de taal te redeneren. Smurf is
	interessant, omdat het een commando heeft om een string als Smurfprogramma
	uit te voeren. We bekijken de gevolgen van dit commando op de semantiek.
\end{abstract}


% vim: set spelllang=nl:
\section{Inleiding}
\label{sec:intro}

Smurf is een esoterische programmeertaal oorspronkelijk ontworpen door Matthew
Westcott. In de specificatie \cite{safalra} beschrijft hij kort wat Smurf is:
\begin{quote}
	Smurf = String-based MURiel Forthoid

	Smurf is a tarpit based on the self-propagation paradigm featured in Muriel.
	The only native data type is the string, and operations are carried out on
	strings in a forty manner.
\end{quote}
We hebben dus te maken met een Forth-achtige programmeertaal. De eigenschappen die we hiervan terugzien in Smurf zijn voornamelijk reflection,
stackgeörienteerd en `geconcateneerd programmeren'. We kunnen het programma dus
dynamisch aanpassen, werken met een stack en schrijven een programma als één
grote functiecompositie (zonder met functieapplicaties te werken). Voordat we
alle commando's bespreken is een voorbeeld op zijn plaats.

\begin{exmp}
	We bekijken het volgende programma:
	\begin{smurf}"papa" "smurf" + o\end{smurf}
	Hier gebruiken we drie functiecomposities om vier functies aan elkaar te
	knopen:
	\begin{itemize}
		\item \smurfinline{"papa"} zet de string `\texttt{papa}' op de stack.
		\item \smurfinline{"smurf"} zet de string `\texttt{smurf}' op de stack.
		\item \smurfinline{+} concateneert de twee elementen bovenop de stack
			(eerst gepushte element eerst) en zet het resultaat op de stack.
		\item \smurfinline{o} output het element bovenop de stack.
	\end{itemize}
	De output van dit programma is dus `papasmurf'.

	We hebben spaties gebruikt voor de leesbaarheid. Dit is toegestaan maar niet
	vereist. Het programma \smurfinline{"papa""smurf"+o} is eveneens geldig.
\end{exmp}

Naast de stack kent Smurf ook een \emph{variable store} die variabelenamen
(strings) naar waardes (strings) stuurt. Het gebruik hiervan is best te
illustreren met een voorbeeld:

\begin{exmp}
	We bekijken het volgende programma:
	\begin{smurf}"smurf" "papa" p "papa" g o\end{smurf}
	Nadat `smurf' en `papa' op de stack zijn gezet gebruiken we \smurfinline{p}
	om de variabele `papa' de waarde `smurf' te geven. Hierna is de stack weer
	leeg. Vervolgens zetten we `papa' op de stack en gebruiken we \smurfinline{g}
	om het bovenste element als variabele op te zoeken in de variable store en de
	waarde ervan op de stack te zetten. Hierbij wordt het bovenste element van de stack verwijdert. De stack bestaat nu dus uit het element
	`smurf'. Met \smurfinline{o} sturen we deze string naar de output.
\end{exmp}

% vim: set spelllang=nl:
\subsection{Commando's}
\label{sec:intro:commands}
We zullen nu kort op informele wijze de verschillende commando's in Smurf
beschrijven. We geven telkens de notatie in Smurf syntax en een leesbaarder
alternatief dat we hieronder zullen gebruiken. Wanneer een commando iets met
elementen op de stack doet, worden die elementen altijd verwijderd.

\begin{description}[style=nextline,font=\normalfont]
	\item[\smurfinline{"..."} of $\StmPush~\texttt{...}$]
		waarbij `\texttt{...}' een string is. Zet de string \texttt{...} op de
		stack.
	\item[\smurfinline{+} of $\StmCat$]
		Concateneert de bovenste twee strings (laagste eerst) op de stack en zet
		het resultaat op de stack.
	\item[\smurfinline{i} of $\StmInput$]
		Plaatst een string van `user input' op de stack. Hierbij wordt
		\texttt{\textbackslash} gebruikt om LF-karakters, dubbele aanhalingstekens
		en backslashes te escapen.
	\item[\smurfinline{o} of $\StmOutput$]
		Stuurt het bovenste element van de stack naar `de output'.
	\item[\smurfinline{p} of $\StmPut$]
		Zet de waarde van de variabelenaam bovenop de stack tot de string daaronder
		op de stack.
	\item[\smurfinline{h} of $\StmHead$]
		Zet het eerste karakter van de string bovenop de stack als string op de
		stack.
	\item[\smurfinline{t} of $\StmTail$]
		Zet alles behalve de head van de string bovenop de stack op de stack.
	\item[\smurfinline{q} of $\StmQuotify$]
		Manipuleert de string bovenop de stack zodat die als $\StmPush$-commando in
		een Smurfprogramma kan worden gebruikt: escapet LF-karakters, dubbele
		aanhalingstekens en backslashes met een backslash, en plaatst dubbele
		aanhalingstekens om de hele string. Het resultaat wordt op de stack gezet.
	\item[\smurfinline{x} of $\StmExec$]
		Voert de string bovenop de stack uit als Smurfprogramma. Het programma
		begint met een schone toestand, d.w.z. met een lege stack en variable
		store. Dit commando maakt condities, recursie en iteratie mogelijk.
\end{description}


% vim: set spelllang=nl:
\subsection{Voorbeelden}
\label{sec:intro:exmp}
Nu we alle commando's hebben gezien bekijken we nog een paar voorbeelden die
aangeven hoe conditionele executie kan worden behaald in Smurf.

% Dit is waarschijnlijk te veel anders.
%
%\begin{exmp}
%	Het volgende programma implementeert een simpel bestelsysteem:
%	\begin{smurf}
%		"ijs" "0" p \\
%		"pizza" "1" p \\
%		"Wil je ijs (0) of pizza (1)?\textbackslash{}n" o \\
%		"Ik ga " i g + " voor je maken!" + o
%	\end{smurf}
%	We zetten de variabelen \verb$0$ en \verb$1$ tot \verb$ijs$ en \verb$pizza$.
%	Vervolgens vragen we de gebruiker een keuze te maken. Met \smurfinline{ig}
%	kan de gebruiker in feite kiezen welke variabele wordt opgehaald.
%
%	De uitvoer kan er dus uitzien als:
%	\begin{verbatim}
%		Wil je ijs (0) of pizza (1)?
%		1
%		Ik ga pizza voor je maken!
%	\end{verbatim}
%\end{exmp}

\begin{exmp}
	\label{exmp:headortail}
	Dit programma laat de gebruiker kiezen welke functie er wordt uitgevoerd:
	\begin{smurf}
		"h""head"p "t""tail"p \\
		"Voer een string in.\textbackslash{}n"o iq \\
		"Wil je de head of de tail?\textbackslash{}n"o \\
		ig +"o"+ x
	\end{smurf}
	Allereerst zetten we de variabelen \verb$head$ en \verb$tail$ tot \verb$h$ en
	\verb$t$, de commando's voor de head en tail van een string. We vragen de
	gebruiker om een string en zetten die in quotes zodat ze gebruikt kan worden
	in een Smurfprogramma. We vragen de gebruiker om `head' of `tail' in te
	voeren en halen de bijbehorende functie op. Op dit moment kan de stack eruit
	zien als \verb$h$, \verb$"mijn string"$ (bovenste element eerst). Met
	\smurfinline{+"o"+} bouwen we hiervan het programma \smurfinline{"mijn
	string"ho}, wat we met \smurfinline{x} uitvoeren. De uitvoer van het
	programma is dus `m', in dit geval.

	Wat gebeurt er als de gebruiker iets anders dan `head' of `tail' invoert? In
	dit geval bestaat er geen variabele met de naam die de gebruiker had
	ingevoerd. Het commando \smurfinline{g} levert dan de lege string op. Dit
	betekent dat we het programma \smurfinline{"mijn string"o} bouwen. Als de
	gebruiker dus geen correcte keuze maakt, dan zal de hele string worden
	teruggegeven.
\end{exmp}

\begin{exmp}
	Het volgende programma is een Smurf interpreter:
	\begin{smurf}
		ix
	\end{smurf}
	We halen input op van de gebruiker en voeren dit uit als Smurfprogramma.
\end{exmp}

Ten slotte geven we nog een wat groter voorbeeld waar recursie wordt
voorgedaan:

\begin{exmp}
	Dit programma geeft als output de input, de tail van de input, de tail van de
	tail van de input, etc., tot en met de lege string.
	\begin{smurf}
		\footnotesize
		i "s"p \\
		"s"g"\textbackslash{}n"+o \\
		"t\textbackslash{}"s\textbackslash{}"p\textbackslash{}"s\textbackslash{}"g\textbackslash{}"\textbackslash{}\textbackslash{}n\textbackslash{}"+o\textbackslash{}"c\textbackslash{}"p\textbackslash{}"c\textbackslash{}"gq\textbackslash{}"s\textbackslash{}"gq\textbackslash{}"c\textbackslash{}"g++q\textbackslash{}"x\textbackslash{}"+\textbackslash{}"s\textbackslash{}"gp\textbackslash{}"\textbackslash{}"\textbackslash{}"\textbackslash{}"p\textbackslash{}"s\textbackslash{}"ggx" "c"p \\
		"c"gq "s"gq "c"g++ \\
		x
	\end{smurf}
	Met de eerste twee regels halen we input op en sturen die naar de output. Ook
	wordt de input in een variabele \verb$s$ opgeslagen.
	De derde regel zet een Smurfprogramma in \verb$c$.
	In de vierde regel plakken we dat gequotifiede programma, de gequotifiede
	input en het programma aan elkaar vast. Het resultaat voeren we uit.

	Op dit moment voeren we dus het programma uit dat bestaat uit twee keer een
	string pushen en vervolgens het programma uit \verb$c$. Dan bekijken we nu
	dit programma, wat we in de derde regel hierboven hebben gedefinieerd, in een
	iets makkelijkere opmaak:
	\begin{smurf}
		t "s"p "s"g "\textbackslash{}n"+o \\
		"c"p \\
		"c"gq "s"gq "c"g++ \\
		q"x"+ "s"gp \\
		""""p \\
		"s"ggx
	\end{smurf}
	Met de eerste regel slaan we de tail van de oude input op in \verb$s$. We
	sturen deze tail ook naar de output. Hierna staat alleen nog de eerste string
	op de stack, d.w.z. het programma waar we naar kijken. Dit slaan we op in
	\verb$c$.
	De derde regel hier is hetzelfde als de vierde regel hierboven: we maken een
	nieuw programma, wat er precies hetzelfde uitziet als dit, behalve dat we van
	de tweede string één karakter hebben weggehaald.

	Het is verleidelijk op dit moment weer \smurfinline{x} te gebruiken. We
	moeten echter eerst nog controleren of we niet met de lege string te maken
	hebben. Het is namelijk niet mogelijk om x uit te voeren als de stack leeg
	is. We maken dus van het nieuwe programma wederom een nieuw programma, wat
	bestaat uit het pushen van het oude programma en vervolgens \smurfinline{x}
	aanroepen. Dit programma slaan we op in de variabele met als naam de waarde
	van \verb$s$ (de input).

	Mocht deze input de lege string zijn, dan overschrijft de vijfde regel dit
	programma met een leeg programma. Vervolgens halen we in de zesde regel het
	programma (of het lege programma, in het geval van de lege string) op, en
	voeren dit uit. Door deze laatste paar regels zullen we dus niet
	\smurfinline{t} uitvoeren op de lege string.

	\medskip
	De uitvoer van dit programma met de input `smurf' is dus:

	\begin{verbatim}
		smurf
		murf
		urf
		rf
		f
	\end{verbatim}
\end{exmp}


% vim: set spelllang=nl:
\subsection{Organisatie} %todo titel

In \autoref{sec:def} beschrijven we formele definities om de semantiek van
Smurf te kunnen specificeren. We kijken naar de syntax, input en output, de
programmatoestand en de transities in de natuurlijke semantiek die we gaan
definiëren. \autoref{sec:rules} beschrijft vervolgens per statement in de
syntax de formele semantiek. Hierbij baseren we ons op de Smurf specificatie
\cite{safalra}, waarbij we dingen verhelderen en ongedefinieerd gedrag
definiëren. In \autoref{sec:anal} bekijken we een stuk code aan de hand van de
gedefinieerde regels.

\autoref{sec:planning} bevat een planning voor het afwerken van het werkstuk,
die uiteindelijk verwijderd zal worden.




% vim: set spelllang=nl:
\section{Definities}
\label{sec:def}

% vim: set spelllang=nl:
\subsection{Syntax}
\label{sec:def:syn}
We definiëren de volgende syntax:
\setlength{\grammarindent}{5em}
\begin{grammar}
	<Pgm> ::= <Stm>:<Pgm> | $\lambda$

	<Stm> ::= `Push' <String>
		\alt `Cat' | `Head' | `Tail' | `Quotify'
		\alt `Put' | `Get'
		\alt `Input' | `Output'
		\alt `Exec'

	<String> ::= <Char><String> | $\lambda$
\end{grammar}

Een karakter, $\SynChar$, is een symbool uit de ASCII tabel.

Programma's zijn lijsten van statements. Merk op dat compositie van statements
hier expliciet is door middel van de \lit{:}, waar compositie in de
oorspronkelijke versie van de syntax impliciet was.

% vim: set spelllang=nl:
\subsection{Input en output}
\label{sec:def:io}

Allereerst definiëren we het type $\Stack{a}$, omdat stacks we veel met stacks
doen in onze semantiek regels. Een $\Stack{a}$ (lees: een stack van elementen
van type $a$) is een simpel datatype met de volgende syntax:

\def\inbrackets#1{$\mathrm{[}#1\mathrm{]}$}
\def\bracka{\inbrackets{a}}
\begin{grammar}
	<Stack \bracka> ::= [<a>:<Stack \bracka>] | `Nil'
\end{grammar}

Op een stack zijn twee instructies gedefinieerd:
\begin{gather*}
	\pushop : a \times \Stack{a} \to \Stack{a} \\
	\push{e}{\stk} \isdef [e:\stk] \\[1em]
	\popop : \Stack{a} \hookrightarrow a \times \Stack{a} \\
	\pop{[e:\stk]} \isdef (e,\stk) \\
\end{gather*}

$\popop$ is een partiële functie omdat $\pop\Nil$ niet gebruikt mag worden in
onze semantiekregels. %todo waarom niet?

\medskip
We zullen de input en output beide als $\Stack{\String}$ modelleren. In feite
zal zelfs blijken dat we op $\Input$ de operatie $\pushop$ niet nodig hebben,
en op $\Output$ de operatie $\popop$ niet zullen gebruiken. Informeel
beschouwen we $\Input$ als een `bron' van $\String$s en $\Output$ als een `put'
van $\String$s. Formeel:
\begin{align*}
	\Input &\isdef \Stack{\String} \\ \Output &\isdef \Stack{\String}
\end{align*}


% vim: set spelllang=nl:
\subsection{Toestanden}
\label{sec:def:state}

Een toestand $s\in\State$ van Smurf bevat zowel een stack als een variable
store. Hieruit volgt de voor de hand liggende definitie voor $\State$,
$$\State \isdef \Stack{\String} \times (\String \to \String),$$
waarbij we $s=(\stk,\str)\in\State$ lezen als de toestand $s$ met stack $\stk$
en variable store $\str$.

De variable store is een totale functie. Initieel zijn alle waardes in de store
$\lambda$ in overeenstemming met de documentatie \cite{safalra}:

\begin{quote}
	Any string can be used as a variable name, including the empty string. All
	variable values are initially set to the empty string.
\end{quote}

Om de waarde van een key $k$ uit store $\str$ te halen gebruiken we simpelweg
$\str~k$. Vervolgens definiëren we $\putop:\SynString \times \SynString \times
(\String\to\String) \to (\String\to\String)$ die gegeven een variabelenaam, een
waarde en een oude store een nieuwe store oplevert:

$$
	\put{\var}{\val}{\str} k =
		\begin{cases}
			\val   & \text{als $k=\var$} \\
			\str~k & \text{als $k\ne\var$}
		\end{cases}
$$


% vim: set spelllang=nl:
\subsection{Transities}
\label{sec:def:trans}
We hebben ervoor gekozen de semantiek van Smurf in natuurlijke semantiek te
definiëren. In principe hadden we er ook voor kunnen kiezen om structurele
operationele semantiek te gebruiken. In het geval van Smurf komt dit ongeveer
op hetzelfde neer. Hoe regels voor Smurf gedefinieerd zouden kunnen worden in
structurele operationele semantiek wordt toegelicht in \autoref{sec:sos}.

Bij het definiëren van de natuurlijke semantiek van Smurf zullen we de
verzameling van transities als een relatie $\to$ tussen
$\Pgm\times\Input\times\State$ en $\Input\times\Output\times\State$ beschouwen.
Dit schrijven we als
$$\trans{\pgm}{\ip}{\st}{\ip'}{\op}{\st'}$$
en lezen we als
\begin{quote}
	``het programma $\pgm$ zal in toestand $\st$ gegeven input $\ip$ leiden tot
	toestand $\st'$, met output $\op$ waarbij $\ip'$ gelijk is aan $\ip$ zonder
	de geconsumeerde input.''
\end{quote}

We hebben het hele programma $\pgm$ nodig voor de pijl, omdat één commando
($\StmExec$) eventuele verdere statements `weggooit'. %todo ander woord
Verder gebruiken we $\Input$ voor $\StmInput$ en hebben we natuurlijk de
$\State$ nodig voor ieder statement: ieder statement verandert de stack, en
sommige statements veranderen bovendien de variable store.




% vim: set spelllang=nl:
\section{Regels} %todo working title
\label{sec:rules}

We zullen nu ieder syntaxelement nader specificeren. Ook zullen regels voor de
natuurlijke semantiek van Smurf worden geïntroduceerd.

De documentatie \cite{safalra} beschrijft niet wat er gebeurt wanneer er niet
genoeg argumenten op de stack staan om een bepaalde instructie uit te voeren.
We kiezen ervoor om het in zulk soort gevallen onmogelijk te maken een
afleidingsboom te maken (in tegenstelling tot bijvoorbeeld een errorstatus aan
de rechterkant van transities toe te voegen), omdat dit het redeneren over
Smurfprogramma's makkelijker zal maken.

% vim: set spelllang=nl:
\subsection{$\lambda$}
\label{sec:rules:lambda}

We hebben één axioma nodig om het basisgeval van het lege programma af te
handelen. Deze regel geeft aan dat het lege programma niets doet: het gebruikt
geen input, geeft geen output, en verandert de state niet.

$$
\begin{prooftree}
	\axjustifies
	\trans
		{\lambda}{\ip}{\st}
		{\ip}{\Nil}{\st}
	\using{\rlambdans}
\end{prooftree}
$$


%% vim: set spelllang=nl:
\subsection{\texttt{Push}}
\label{sec:rules:push}

\begin{quote}
	"text" - Places the string on top of the stack (without the quotes). The
	string may include the following escape sequences:
	
	\verb$\n$ = newline \\
	\verb$\"$ = the \verb$"$ character \\
	\verb$\\$ = the \verb$\$ character.
\end{quote}

De string tussen de aanhalingstekens word op de stack gezet, nadat escape
sequences eruit zijn gehaald middels de hulpfunctie $\unescapeop$. Het is,
zoals het commentaar op de specificatie \cite{safalra} aangeeft, niet
gedefinieerd wat er met ongeldige escape sequences gebeurt.

Dit geeft de volgende regel:

$$
\begin{prooftree}
	\trans
		{\pgm}{\ip}{(\push{s}{\stk}), \str)}
		{\ip'}{\op}{\st}
	\justifies
	\trans
		{\StmPush~s:\pgm}{\ip}{(\stk,\str)}
		{\ip'}{\op}{\st}
	\using{\rpushns}
\end{prooftree}
$$

De definitie van $\unescapeop$ is als volgt:

$$
	\unescape c =
		\begin{cases}
			\text{het LF-karakter}      & \text{als $c=\texttt{n}$} \\
			\texttt{"}                  & \text{als $c=\texttt{"}$} \\
			\texttt{\textbackslash}     & \text{als $c=\texttt{\textbackslash}$} \\
			\texttt{\textbackslash~$c$} & \text{anderszins}
		\end{cases}
$$

Het laatste alternatief geeft aan dat `ongeldige escape sequences' worden
behandeld alsof de backslash er twee keer stond. Dit is in overeenstemming met
het commentaar op de specificatie en met de Perl interpreter: %todo referentie
\begin{quote}
	This [the specification] does not specify the behaviour of invalid escape
	sequences. The Perl interpreter treats invalid escape sequences as if the
	backslash had occured twice - that is, \textbackslash X is treated as
	\textbackslash\textbackslash X. For maximum compatibility, Smurf programs
	should not rely on this behaviour and should always ensure valid escape
	sequences are used.
\end{quote}
 

% vim: set spelllang=nl:
\subsection{\texttt{Head}}
\label{sec:rules:head}

\begin{quote}
	h - Pops a string from the stack, and pushes its head, ie [sic] the first
	character. This causes an error if used on the empty string.
\end{quote}

In plaats van het geven van een error kiezen we ervoor te voorkomen dat we een
afleidingsboom kunnen maken wanneer $\StmHead$ wordt uitgevoerd op het moment
dat het element bovenop de stack de lege string is.

Dit geeft de volgende regel:
\therheadns%

% vim: set spelllang=nl:
\subsection{\texttt{Tail}}
\label{sec:rules:tail}

\begin{quote}
    t - Pops a string from the stack, and pushes its tail, ie all but the first character. This causes an error if used on the empty string.
\end{quote}

In plaats van het geven van een error kiezen we ervoor te voorkomen dat we een
afleidingsboom kunnen maken wanneer $\StmTail$ wordt uitgevoerd op het moment
dat het element bovenop de stack de lege string is.

Dit geeft de volgende regel:
\thertailns%

%% vim: set spelllang=nl:
\subsection{\texttt{Quotify}}
\label{sec:rules:quotify}

\begin{quote}
	q - "Quotifies" the string on top of the stack, so that it can be placed into
	a Smurf program as a literal string, eg \texttt{Arthur "two-sheds" Jackson}
	becomes \texttt{"Arthur $\backslash$"two-sheds$\backslash$" Jackson"}.
\end{quote}


Er worden aanhalingstekens om de string bovenop de stack gezet. Als er in de
oorspronkelijke string aanhalingstekens, backslashes of LF-karakters staan, dan
wordt hier een \verb$\$ voor geplaatst. Hiervoor gebruiken we de hulpfunctie
$\escapeop$.

Dit geeft de volgende regel:
$$
\begin{prooftree}
	\trans
		{\pgm}{\ip}{(\push{\texttt{"}\escape{s}\texttt{"}}{\stk'}, \str) }
		{\ip'}{\op}{\st}
	\justifies
	\trans
		{\StmQuotify:\pgm}{\ip}{(\stk,\str)}
		{\ip'}{\op}{\st}
	\using{\rquotifyns}
	\qquad
	\text{met $\pop{\stk} = (s, stk')$.}
\end{prooftree}
$$

$$
	\escape{c~s} =
		\begin{cases}
			\texttt{\textbackslash}~c~\escape{s} & \text{als
				$c\in\{\texttt{"},\texttt{\textbackslash},\text{het LF-karakter}\}$} \\
			c~\escape{s}                & \text{anderszins}
		\end{cases}
$$

%% vim: set spelllang=nl:
\subsection{\texttt{Cat}}
\label{sec:rules:cat}

\begin{quote}
    + - concatenates the top two strings on the stack. The string pushed earlier
		appears earlier in the resulting string, eg \smurfinline{"Zork" "mid" +}
		would result in the string \texttt{Zorkmid} being placed on the stack.
\end{quote}

De string bovenop de stack wordt toegevoegd aan de string hieronder. Het
resultaat wordt op de stack gezet.

Dit geeft de volgende regel:

$$
\begin{prooftree}
	\trans
		{\pgm}{\ip}{(\push{s1~s2}{\stk''}, \str)}
		{\ip'}{\op}{\st}
	\justifies
	\trans
		{\StmCat:\pgm}{\ip}{(\stk,\str)}
		{\ip'}{\op}{\st}
	\using{\rcatns}
	\qquad
	\text{met\enspace
	\parbox{36mm}{$\pop{\stk} = (s_2,\stk') $,\\$ \pop{\stk'} = (s_1,\stk'')$.}}
\end{prooftree}
$$


% vim: set spelllang=nl:
\subsection{\texttt{Get}}
\label{sec:rules:exec}

\begin{quote}
	g - Pops a variable name from the stack, and pushes the value of the
	variable.
\end{quote}

De regel voor dit statement spreekt voor zich. In het geval dat $\stk$ leeg is,
is $\pop\stk$ niet gedefinieerd en kunnen we de regel dus niet toepassen. Omdat
$\StmGet$ geen IO gebruikt kunnen we $\ip$, $\ip'$ en $\op$ direct doorgeven.

$$
\begin{prooftree}
	\trans
		{\pgm}{\ip}{(\push{\str~\var}{\stk'}, \str)}
		{\ip'}{\op}{\st}
	\justifies
	\trans
		{\StmGet:\pgm}{\ip}{(\stk,\str)}
		{\ip'}{\op}{\st}
	\using{\rgetns}
	\qquad
	\text{met $(\var,\stk') = \pop{\stk}$.}
\end{prooftree}
$$


% vim: set spelllang=nl:
\subsection{\texttt{Put}}
\label{sec:rules:put}

\begin{quote}
	p - Pops a variable name from the stack, pops a value from the stack, and
	assigns that value to the variable name.
\end{quote}

We halen twee strings van de stack en gebruiken $\putop$ om een nieuwe variable
store te krijgen. Hiermee wordt de rest van het programma uitgevoerd. Als er
minder dan twee elementen op de stack staan kan deze regel niet worden
toegepast, aangezien $\popop$ een partiële functie is.

$$
\begin{prooftree}
	\trans
		{\pgm}{\ip}{(\stk'', \put\var\val\str)}
		{\ip'}{\op}{\st}
	\justifies
	\trans
		{\StmPut:\pgm}{\ip}{(\stk,\str)}
		{\ip'}{\op}{\st}
	\using{\rputns}
	\qquad
	\text{met\enspace
		\parbox{36mm}{$\pop{\stk} = (\var,\stk')$,
		\\$\pop{\stk'}= (\val,\stk'')$.}}
\end{prooftree}
$$


% vim: set spelllang=nl:
\subsection{\texttt{Input}}
\label{sec:rules:input}

\begin{quote}
	i - takes a string from user input, and places it on the stack.
\end{quote}

Dit commando is erg slecht gedefinieerd. Mag deze input
bijvoorbeeld line feeds bevatten? Een naïeve implementatie die van stdin
gebruik maakt zou standaard splitsen op line feeds en dit dus niet toestaan.
Een andere implementatie zou gebruik kunnen maken van grafische prompts waarin
het wél mogelijk is meerdere regels input te geven.

Om bij het redeneren over de semantiek geen last te hebben van zulk soort
implementatiedetails hebben we ervoor gekozen hier tot op een hoog niveau van
te abstraheren. We gaan ervan uit dat we een stack van strings, $\Input$,
hebben die ons de input geeft. Hoe deze stack is geïmplementeerd doet niet ter
zake. Ook leggen we geen restricties op aan de karakters waaruit haar elementen
bestaan.

We gaan er verder van uit dat de gebruiker zijn programma volledig wil
uitvoeren en dus voldoende input zal geven, waardoor $\pop\ip$ altijd
gedefinieerd is. Geeft de gebruiker niet genoeg input, dan hoeven we dus geen
afleidingsboom te maken (en we zullen dit ook onmogelijk maken).

Dit geeft de volgende regel:

$$
\begin{prooftree}
	\trans
		{\pgm}{\ip'}{(\push\val\stk, \str)}
		{\ip''}{\op}{\st}
	\justifies
	\trans
		{\StmInput:\pgm}{\ip}{(\stk,\str)}
		{\ip''}{\op}{\st}
	\using{\rinputns}
	\qquad
	\text{met $\pop\ip = (\val,\ip')$.}
\end{prooftree}
$$


%% vim: set spelllang=nl:
\subsection{\texttt{Output}}

\begin{quote}
	o - Output the string at the top of the stack
\end{quote}

Net als bij het inputcommando gaan we op een abstracte wijze met de output om.
We houden gedurende het hele programma een stack van strings, genaamd $\Output$
bij waar het programma zijn output naar wegschrijft.

Dit geeft de volgende regel:
\theroutputns%

waarbij $\op$ in de bovenste regel de gehele outputstack weergeeft. Merk op
dat eenzelfde regel waar $s$ niet voor op de stack wordt gezet maar achter,
even geldig is. Geen van beide opties is beter dan de ander omdat we geen
aannames doen over hoe de $\Output$-stack wordt verwerkt.

% vim: set spelllang=nl:
\subsection{\texttt{Exec}}
\label{sec:rules:exec}

\begin{quote}
	x - Executes the string at the top of the stack as a Smurf program. The stack
	and variable store are erased.
\end{quote}

We halen een string van de stack en gebruiken $\parsepgmop$ om dit in een
programma om te zetten. Hieronder zullen we $\parsepgmop$ definiëren. Dit wordt
het nieuwe programma om uitgevoerd te worden.  Als de $\stk$ leeg is is deze
regel niet toepasbaar, omdat $\pop\stk$ dan niet gedefinieerd is. Ook is deze
regel niet toepasbaar als de gepopte string zelf geen geldig Smurf-programma
is, omdat $\parsepgmop$ dan niet gedefinieerd is.

$$
\begin{prooftree}
	\trans
		{\pgm'}{\ip}{(\Nil, \emptystore)}
		{\ip'}{\op}{\st}
	\justifies
	\trans
		{\StmExec:\pgm}{\ip}{(\stk,\str)}
		{\ip'}{\op}{\st}
	\using{\rexecns}
	\qquad
	\text{met\enspace
		\parbox{36mm}{$ \pop{\stk} =(\var,\stk')$,\\
			$\pgm' = \parsepgm{\var'}$.}
	}
\end{prooftree}
$$

\medskip
$\parsepgmop$ definiëren we als volgt, met een hulpfunctie $\parsestrop$:

$$
	\parsepgm s =
		\begin{cases}
			\lambda                   & \text{als $s=\lambda$}\\
			\parsepgm{s'}             & \text{als $s=c~s'$ met $c$ whitespace}\\
			\parsestr{s'}             & \text{als $s=\texttt{"}~s'$} \\
			\StmCat:\parsepgm{s'}     & \text{als $s=\texttt{+}~s'$} \\
			\StmHead:\parsepgm{s'}    & \text{als $s=\texttt{h}~s'$} \\
			\StmTail:\parsepgm{s'}    & \text{als $s=\texttt{t}~s'$} \\
			\StmQuotify:\parsepgm{s'} & \text{als $s=\texttt{q}~s'$} \\
			\StmPut:\parsepgm{s'}     & \text{als $s=\texttt{p}~s'$} \\
			\StmGet:\parsepgm{s'}     & \text{als $s=\texttt{g}~s'$} \\
			\StmInput:\parsepgm{s'}   & \text{als $s=\texttt{i}~s'$} \\
			\StmOutput:\parsepgm{s'}  & \text{als $s=\texttt{o}~s'$} \\
			\StmExec:\parsepgm{s'}    & \text{als $s=\texttt{x}~s'$} \\
		\end{cases}
$$

Het tweede geval van $\parsepgmop$ zorgt ervoor dat een programma-string
bijvoorbeeld spaties mag bevatten, die syntactisch zelf geen betekenis hebben.
Dit is in overeenkomst met de specificatie, maar op zich niet nodig.

$$
	\parsestr s =
		\begin{cases}
			\lambda:\parsepgm{s'} & \text{als $s=\texttt{"}~s'$} \\
			\unescape{c}~\parsestr{s'} & \text{als $s=\texttt{\textbackslash}~c~s'$
				met $c \in\Char$} \\
			c~\parsestr{s'} & \text{als $s=c~s'$ met $c
				\in\Char\setminus\{\texttt{"},\texttt{\textbackslash}\}$}\\
		\end{cases}
$$

Het tweede geval van $\parsestrop$ zorgt ervoor dat ge-escapete
aanhalingstekens de string niet beëindigen. Hierbij gebruiken we $\unescapeop$
om bepaalde karakters te unescapen. Zie voor de definitie van deze regel
\autoref{sec:rules:push}.




% vim:set spelllang=nl:
\section{Smurf in structurele operationele semantiek}
\label{sec:sos}
We willen ook nog graag laten zien hoe de definities van Smurf eruit zien als
je de structurele operationele semantiek gebruikt. In principe maakt het voor
de analyse van Smurf niet uit of je natuurlijke semantiek gebruikt of
structurele operationele semantiek. Omdat natuurlijke semantiek en structurele
operationele semantiek equivalent zijn wanneer de regels goed gedefinieerd
zijn. Wij hadden een voorkeur om natuurlijke semantiek te gebruiken omdat we
hier meer bekend mee zijn. Echter willen we ook nog graag laten zien hoe het
eruit zou zien als je structurele operationele semantiek zou gebruiken.

We gebruiken hier in plaats van de labdaregel twee compositieregels om hiervan
ook het verschil aan te geven.

De twee compositieregels zijn als volgt:

$$
\begin{prooftree}
	\sostrans
		{\stm}{\ip}{\op}{(\stk, \str)}
		{\pgm'}{\ip'}{\op'}{(\stk', \str')}
	\justifies
	\sostrans
		{\stm:\pgm}{\ip}{\op}{(\stk,\str)}
		{\pgm'}{\ip'}{\op'}{(\stk',\str')}
	\using{\rcompeensos}
	\qquad
\end{prooftree}
$$

\medskip
$$
\begin{prooftree}
	\sostranseind
		{\stm}{\ip}{\op}{(\stk, \str)}
		{\ip'}{\op'}{(\stk', \str')}
	\justifies
	\sostrans
		{\stm:\pgm}{\ip}{\op}{(\stk,\str)}
		{\pgm}{\ip'}{\op'}{(\stk',\str')}
	\using{\rcomptweesos}
	\qquad
\end{prooftree}
$$

\bigskip
De regel voor $\StmTail$ zou in de structurele operationele semantiek als volgt
zijn:

$$
\begin{prooftree}
	\axjustifies
	\sostranseind
		{\StmTail}{\ip}{\op}{(\stk,\str)}
		{\ip}{\op}{(\push{s}{\stk'}, \str)}
	\using{\rtailsos}
	\qquad
	\text{met $\pop{\stk} = (c~s,\stk')$.}
\end{prooftree}
$$

% vim: set spelllang=nl:
\section{Analyse}
\label{sec:analyse}

Als analyse willen we graag een stuk code dat een string omdraait bekijken aan
de hand van onze semantiekregels. Deze code ziet er als volgt uit:

\begin{smurf}
	\footnotesize
	"+"i+ ""p ""gtg ""gt "i"p\\
	"\textbackslash{}"\textbackslash{}"p\textbackslash{}"i\textbackslash{}"gh\textbackslash{}"o\textbackslash{}"g+\textbackslash{}"o\textbackslash{}"p\textbackslash{}"i\textbackslash{}"gt\textbackslash{}"i\textbackslash{}"p\textbackslash{}"\textbackslash{}\textbackslash{}\textbackslash{}"+\textbackslash{}\textbackslash{}\textbackslash{}"\textbackslash{}\textbackslash{}\textbackslash{}\textbackslash{}\textbackslash{}\textbackslash{}"\textbackslash{}\textbackslash{}\textbackslash{}"p\textbackslash{}"\textbackslash{}"i\textbackslash{}"gq+\textbackslash{}"tg\textbackslash{}"+\textbackslash{}"i\textbackslash{}"gq+\textbackslash{}"\textbackslash{}\textbackslash{}\\
    \textbackslash{}"i\textbackslash{}\textbackslash{}\textbackslash{}"p\textbackslash{}"+\textbackslash{}"o\textbackslash{}"gq+\textbackslash{}"\textbackslash{}\textbackslash{}\textbackslash{}"o\textbackslash{}\textbackslash{}\textbackslash{}"p\textbackslash{}"+\textbackslash{}"\textbackslash{}"gq+\textbackslash{}"\textbackslash{}"g+\textbackslash{}"\textbackslash{}"p\textbackslash{}"o\textbackslash{}"gq\textbackslash{}"o\textbackslash{}"+\textbackslash{}"+\textbackslash{}"pgx"\\
	""p "\textbackslash{}"+\textbackslash{}"\textbackslash{}"\textbackslash{}"p" "i"gq+ "tg"+ "i"gq+\\
	"\textbackslash{}"i\textbackslash{}"p\textbackslash{}"\textbackslash{}""+ ""gq+ ""g+ ""p "" "+" "i"g+ pgx
\end{smurf}

Omgezet naar onze leesbare (hier een relatief begrip) syntax ziet de code er zo
uit:

%todo leesbaarheid betekent ook witregels waar logisch
\begin{smurf}
	\footnotesize
	
	\StmPush~"+"~: 
	\StmInput~: 
	\StmCat~: 
	\StmPush~$\lambda$~: 
	\StmPut~: 
	\StmPush~$\lambda$~: 
	\StmGet~: 
	\StmTail~:\\
	\StmGet~: 
	\StmPush~$\lambda$~: 
	\StmGet~: 
	\StmTail~: 
	\StmPush~"i"~: 
	\StmPut~:\\
	\StmPush~"\textbackslash{}"\textbackslash{}"p\textbackslash{}"i\textbackslash{}"gh\textbackslash{}"o\textbackslash{}"g+\textbackslash{}"o\textbackslash{}"p\textbackslash{}"i\textbackslash{}"gt\textbackslash{}"i\textbackslash{}"p\textbackslash{}"\textbackslash{}\textbackslash{}\textbackslash{}"+\textbackslash{}\textbackslash{}\textbackslash{}"\textbackslash{}\textbackslash{}\textbackslash{}"\textbackslash{}\textbackslash{}\textbackslash{}"p\textbackslash{}"\textbackslash{}"i\textbackslash{}"gq+\textbackslash{}"tg\textbackslash{}"+\textbackslash{}"i\textbackslash{}"gq+\textbackslash{}"\textbackslash{}\textbackslash{}\\
    \textbackslash{}"i\textbackslash{}\textbackslash{}\textbackslash{}"p\textbackslash{}"+\textbackslash{}"o\textbackslash{}"gq+\textbackslash{}"\textbackslash{}\textbackslash{}\textbackslash{}"o\textbackslash{}\textbackslash{}\textbackslash{}"p\textbackslash{}"+\textbackslash{}"\textbackslash{}"gq+\textbackslash{}"\textbackslash{}"g+\textbackslash{}"\textbackslash{}"p\textbackslash{}"o\textbackslash{}"gq\textbackslash{}"o\textbackslash{}"+\textbackslash{}"+\textbackslash{}"pgx"~:\\
	\StmPush~$\lambda$~: 
	\StmPut~: 
	\StmPush~"\textbackslash{}"+\textbackslash{}"\textbackslash{}"\textbackslash{}"p"~:\\
	\StmPush~"i"~: 
	\StmGet~: 
	\StmQuotify~: 
	\StmCat~: 
	\StmPush~"tg"~: 
	\StmCat~: 
	\StmPush~"i"~: 
	\StmGet~: 
	\StmQuotify~: 
	\StmCat~: 
	\StmPush~"\textbackslash{}"i\textbackslash{}"p\textbackslash{}"\textbackslash{}""~: 
	\StmCat~: 
	\StmPush~$\lambda$~: 
	\StmGet~: 
	\StmQuotify~: 
	\StmCat~: 
	\StmPush~$\lambda$~: 
	\StmGet~: 
	\StmCat~: 
	\StmPush~$\lambda$~: 
	\StmPut~: 
	\StmPush~$\lambda$~: 
	\StmPush~"+"~: 
	\StmPush~"i"~: 
	\StmGet~: 
	\StmCat~: 
	\StmPut~: 
	\StmGet~: 
	\StmExec~: 
    $\lambda$

    
\end{smurf}

Nu laten we zien dat deze code daadwerkelijk alle mogelijke strings omdraait,
oftewel: er is een afleidingsboom voor
$$
\trans
	{Programma}{[s:\Nil]}{(\Nil,\emptystore)}
	{\Nil}{s^R}{\st}
$$
voor alle s, waar
$$(c~s)^R=s^R c$$
$$\lambda^R=\lambda$$

\begin{proof}[Bewijs]
	Met inductie naar de lengte van $s$.

	Basisgeval: $s=\lambda$.

	$$%Todo fix dit
	\begin{prooftree}
		\[
			\[
				\[
					\[
						\vdots %todo afmaken
						\justifies
						\trans
							{\StmPut : \StmPush~\lambda : ...~}{\Nil}{([\lambda:["+":\Nil]], \emptystore)}
							{\Nil}{\lambda}{\st}
						\using{\rputns}
					\]
					\justifies
					\trans
						{\StmPush~\lambda : \StmPut : ...~}{\Nil}{(["+":\Nil], \emptystore)}
						{\Nil}{\lambda}{\st}
					\using{\rpushns}
				\]
				\justifies
				\trans
					{\StmCat : \StmPush~\lambda : ...~}{\Nil}{([\lambda:["+":\Nil]], \emptystore)}
					{\Nil}{\lambda}{\st}
				\using{\rcatns}
			\]
			\justifies
			\trans
				{\StmInput : \StmCat : ...~}{[\lambda:\Nil]}{(["+":\Nil], \emptystore)}
				{\Nil}{\lambda}{\st}
			\using{\rinputns}
		\]
		\justifies
		\trans
			{\StmPush~"+" : \StmInput : ...~}{[\lambda:\Nil]}{(\Nil,\emptystore)}
			{\Nil}{\lambda}{\st}
		\using{\rpushns}
	\end{prooftree}
	$$

	%todo afmaken
\end{proof}

% vim: set spelllang=nl:
\section{CleanSmurf}
\label{sec:cleansmurf}

Semantiekregels vertalen zich uiterst gemakkelijk naar een implementatie van
een interpreter in een functionele taal. \emph{CleanSmurf}~\cite{cleansmurf} is
een interpreter voor Smurf, geschreven in Clean, dat onze semantiekregels
volgt. Omdat het de semantiekregels volgt, was het niet lastig dit uit te
breiden naar een programma dat een afleidingsboom genereert. In dit hoofdstuk
beschrijven we de globale opzet van dit programma.

\subsection{Types}
\label{sec:cleansmurf:types}
Het programma houdt de expressieve syntax uit \autoref{sec:def:syn} aan. Het
type \CI{Stm} is dus:

\lstinputlisting[firstline=15,lastline=20]{CleanSmurf/Smurf.dcl}

We gebruiken lijsten als stacks en implementeren de store met behulp van een
tabel van naam-waarde paren:

\lstinputlisting[firstline=22,lastline=27]{CleanSmurf/Smurf.dcl}

Ten slotte definiëren we transities. Aangezien alle semantiekregels hooguit één
premisse hebben, kunnen we een afleidingsboom als lijstje transities zien:

\lstinputlisting[firstline=35,lastline=36]{CleanSmurf/Smurf.dcl}

Met deze definities kunnen we een \CI{step :: Program State -> Maybe (Program,
State)} definiëren. Dit komt bijna overeen met de structuur van transities. We
moeten een \CI{Program} teruggeven, omdat we slechts één stap zetten. Wat dit
betreft lijken de regels van \emph{CleanSmurf} meer op regels in structurele
operationele semantiek. We hebben al laten zien dat deze regels erg veel lijken
op die in natuurlijke semantiek.

Het type van \CI{step} neemt nog geen Input/Output mee. We maken er een
overloaded functie van, zodat hij voor meerdere inputmethodes kan worden
gebruikt. Het uiteindelijke type is:

\lstinputlisting[firstline=56,lastline=56]{CleanSmurf/Smurf.dcl}

Hoe dit precies werkt is niet belangrijk voor de beschrijving hier. Het derde
argument, van type \CI{io}, is de `IO-toestand'. Het vierde argument, van type
\CI{IO io}, omvat een input-functie die strings oplevert (gegeven de
IO-toestand) en een output-functie die strings opneemt (in de IO-toestand).

\medskip
Verder definiëren we een aantal hulpfuncties:

\lstinputlisting[firstline=232,lastline=245]{CleanSmurf/Smurf.icl}

De partiële functies zijn hier gesimuleerd met het \CI{Maybe}-type. Dit laat
ons monadische operatoren gebruiken in het uitwerken van de interpreter.

\subsection{Semantiekregels}
\label{sec:cleansmurf:regels}
Iedere semantiekregel vertaalt min of meer direct naar een functiealternatief
voor \CI{step}. Echter, omdat we geen \CI{run}- maar een \CI{step}-functie
schrijven, moeten we compositie expliciet maken.

Als voorbeeld zullen we de implementatie van $\StmHead$ bekijken. Aangezien
\CI{pop} en \CI{head} allebei een \CI{Maybe} opleveren, kunnen we de resultaten
gemakkelijk binden. Vervolgens wordt de stack geüpdate en wordt de rest van het
programma (\CI{p}) teruggegeven. De IO-toestand wordt zonder gebruik
doorgegeven. Het vierde argument hebben we niet nodig en kan dus worden
genegeerd.

\lstinputlisting[firstline=213,lastline=215]{CleanSmurf/Smurf.icl}

Andere regels, voor $\StmPush$, $\StmTail$, $\StmCat$, $\StmQuotify$ en ook
$\StmPut$ en $\StmGet$ gaan op soortgelijke wijze. Ook $\StmInput$ en
$\StmOutput$ kunnen op dezelfde manier worden geschreven, afgezien van het feit
dat de IO-toestand en -functies gebruikt moeten worden.

In het geval van $\StmExec$ kunnen we handig gebruik maken van het feit dat
\CI{step} een nieuw programma oplevert:

\lstinputlisting[firstline=228,lastline=230]{CleanSmurf/Smurf.icl}

Zoals te zien wordt een nieuwe toestand gemaakt (\CI{zero}) waarin dit nieuwe
programma (\CI{p}) wordt uitgevoerd. De compositie \CI{parse o fromString}
parseert een String van de stack en levert een \CI{Maybe Program} op. Dus ook
deze wat afwijkende regel levert geen problemen op.

\subsection{Afleidingsbomen}
Door herhaaldelijk gebruik te maken van \CI{step} kunnen we gemakkelijk een
afleidingsboom genereren. We maken hierbij gebruik van een aantal handige
eigenschappen van afleidingsbomen voor Smurf:

\begin{itemize}
	\item Alle semantiekregels hebben ten hoogste één premisse, waardoor we een
		boom als lijst van transities kunnen representeren.
	\item Ieder commando heeft precies één regel. Hierdoor is aan het eerste
		statement van een programma te herkennen welke regel wordt toegepast. Dit
		hoeven we dus niet in de types \CI{Transition} en \CI{DerivationTree} op te
		slaan.
	\item Doordat condities van de semantiekregels enkel afhangen van de
		linkerkant van de conclusie (het programma, de input en de toestand), en
		deze informatie beschikbaar is op het moment dat de boom gemaakt wordt,
		hoeven we geen backtracking toe te passen. Wanneer het niet lukt de regel
		die bij het eerste commando van het programma hoort toe te passen, weten we
		zeker dat er geen afleidingsboom bestaat.
\end{itemize}

De functie \CI{tree} is als volgt geïmplementeerd:

\lstinputlisting[firstline=274]{CleanSmurf/Smurf.icl}

Bij het genereren van bomen leggen we de input van te voren vast in een lijst.
We slaan de output op in een andere lijst. Dit is wat het type \CI{ListIO}
inhoudt --- de implementatie hiervan is irrelevant hier.

Het eerste dat \CI{tree} doet is het berekenen van de volgende stap. Lukt dat
niet (\CI{isNothing mbPgmSt}), dan kunnen we ook geen boom maken.

Omdat \CI{step} voor een leeg programma een nieuw leeg programma oplevert,
moeten we in deze functie controleren of het nieuwe programma leeg is. Is dit
het geval, dan zetten we de $\lambda$-regel zelf in de boom als \CI{([],
io.input, st) --> (io.input, [], st)}.

Is dit niet het geval, dan maken we recursief een boom voor de premisse
(\CI{tree pgm st io iof}). Lukt dit niet, dan kunnen we geen boom maken.
Anders, dan voegen we de boom en de transitie gevonden met \CI{step} samen.

% vim: set spelllang=nl:
\section{Planning}
\label{sec:planning}

\begin{tabular}{ l l l }
  week & taken & deadlines \\
  17 &  & 17 april: inleveren eerste versie werkstuk\\
  18 & voorbereiden voortgangsgesprek &  \\
  19 & verwerken feedback gesprek 1 & 9 mei of 11 mei: voortgangsgesprek 1\\
  20 & start analyse en verwerking feedback gesprek 1 (inleiding en definities liggen nu vast) & \\
  21 & verdere uitwerking analyse (Regels liggen nu vast)& 27 mei: inleveren tweede versie werkstuk\\
  22 & verwerken feedback gesprek 2 & voorgangsgesprek 2\\
  23 & uitdiepen van de analyse & \\
  24 & finishing toutch (analyse ligt nu vast)& 17 juni: inleveren definitieve versie werkstuk\\
\end{tabular}

\begin{thebibliography}{9}
	\bibitem{safalra} Smurf Specification,
	\url{http://safalra.com/programming/esoteric-languages/smurf/specification/}.
	Opgehaald 27 april 2016.

	\bibitem{esolang:prog} Smurf --- Reversing input on Esolang,
	\url{http://esolangs.org/wiki/Smurf#Reversing_input}. Opgehaald 27 mei 2016.

	\bibitem{cleansmurf} \emph{CleanSmurf} --- Smurf interpreter in Clean,
	\url{https://github.com/camilstaps/CleanSmurf}. Opgehaald 10 juni 2016.
\end{thebibliography}



\appendix
% vim: set spelllang=nl:
\section{Afleidingsbomen}
\label{sec:app:trees}

\def\makecaption#1#2{%
	\thispagestyle{empty}%
	\null\vfill\captionof{figure}{#1\label{#2}}
}

\begin{landscape}
	\includepdf[landscape,pagecommand={\makecaption{Afleidingsboom voor het
	buitenste programma voor de string $\lambda$}{fig:tree:lambda}}]{tree-gen-lambda.pdf}
	\includepdf[landscape,pagecommand={\makecaption{Afleidingsboom voor het buitenste programma voor strings met lengte $n\ge1$}{fig:tree:bootstrap}}]{tree-gen-bootstrap.pdf}
	\includepdf[landscape,pagecommand={\makecaption{Afleidingsboom voor het binnenste programma voor strings met lengte $1$}{fig:tree:bootstrap-base}}]{tree-gen-bootstrap-base.pdf}
	\includepdf[landscape,pagecommand={\makecaption{Afleidingsboom voor het binnenste programma voor strings met lengte $n>1$}{fig:tree:bootstrap-step}}]{tree-gen-bootstrap-step.pdf}
\end{landscape}


\end{document}
