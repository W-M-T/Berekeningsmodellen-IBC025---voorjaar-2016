% vim: set spelllang=nl:
\section{Inleiding}
\label{sec:intro}

Smurf is een esoterische programmeertaal oorspronkelijk ontworpen door Matthew
Westcott. In de specificatie \cite{safalra} beschrijft hij kort wat Smurf is:
\begin{quote}
	Smurf = String-based MURiel Forthoid

	Smurf is a tarpit based on the self-propagation paradigm featured in Muriel.
	The only native data type is the string, and operations are carried out on
	strings in a forty manner.
\end{quote}
We hebben dus te maken met een Forth-achtige programmeertaal. De eigenschappen
hiervan die we terugzien in Smurf zijn voornamelijk reflection,
stackgeörienteerd en `geconcateneerd programmeren'. We kunnen het programma dus
dynamisch aanpassen, werken met een stack en schrijven een programma als één
grote functiecompositie (zonder met functieapplicaties te werken). Voordat we
alle commando's bespreken is een voorbeeld op zijn plaats.

\begin{exmp}
	We bekijken het volgende programma:
	\begin{smurf}"papa" "smurf" + o\end{smurf}
	Hier gebruiken we drie functiecomposities om vier functies aan elkaar te
	knopen:
	\begin{itemize}
		\item \smurfinline{"papa"} zet de string `\texttt{papa}' op de stack.
		\item \smurfinline{"smurf"} zet de string `\texttt{smurf}' op de stack.
		\item \smurfinline{+} concateneert de twee elementen bovenop de stack
			(eerst gepushte element eerst) en zet het resultaat op de stack.
		\item \smurfinline{o} output het element bovenop de stack.
	\end{itemize}
	De output van dit programma is dus `papasmurf'.

	We hebben spaties gebruikt voor de leesbaarheid. Dit is toegestaan maar niet
	vereist. Het programma \smurfinline{"papa""smurf"+o} is eveneens geldig.
\end{exmp}

Naast de stack kent Smurf ook een \emph{variable store} die variabelenamen
(strings) naar waardes (strings) stuurt. Het gebruik hiervan is best te
illustreren met een voorbeeld:

\begin{exmp}
	We bekijken het volgende programma:
	\begin{smurf}"smurf" "papa" p "papa" g o\end{smurf}
	Nadat `smurf' en `papa' op de stack zijn gezet gebruiken we \smurfinline{p}
	om de variabele `papa' de waarde `smurf' te geven. Hierna is de stack weer
	leeg. Vervolgens zetten we `papa' op de stack en gebruiken we \smurfinline{g}
	om het bovenste element als variabele op te zoeken in de variable store en de
	waarde ervan op de stack te zetten. De stack bestaat nu dus uit het element
	`smurf'. Met \smurfinline{o} sturen we deze string naar de output.
\end{exmp}

% vim: set spelllang=nl:
\subsection{Commando's}
\label{sec:intro:commands}
We zullen nu kort op informele wijze de verschillende commando's in Smurf
beschrijven. We geven telkens de notatie in Smurf syntax en een leesbaarder
alternatief dat we in de rest van dit werkstuk zullen gebruiken. De commando's
zijn dus niet exact hetzelfde als in de voorbeeldprogramma's en de specificatie
van de taal~\cite{safalra}. Met onze leesbaardere variant is het makkelijker om
over de taal te redeneren.

Voor ieder commando geldt dat wanneer het elementen op de stack gebruikt, die
elementen worden verwijderd.

\begin{description}[style=nextline,font=\normalfont]
	\item[\smurfinline{"..."} of $\StmPush~\texttt{...}$, waar \lit{...} een
		string is]
		Zet de string \lit{...} op de stack.
	\item[\smurfinline{+} of $\StmCat$]
		Concateneert de bovenste twee strings (laagste eerst) op de stack en zet
		het resultaat op de stack.
	\item[\smurfinline{i} of $\StmInput$]
		Plaatst een string van `user input' op de stack. Hierbij wordt
		\lit{\textbackslash} gebruikt om LF-karakters, dubbele aanhalingstekens en
		backslashes te escapen. Afhankelijk van de inputmethode kan het ook
		mogelijk zijn dat LF-karakters in de string voorkomen. In deze specificatie
		abstraheren we van inputmethodes.
	\item[\smurfinline{o} of $\StmOutput$]
		Stuurt het bovenste element van de stack naar `de output'.
	\item[\smurfinline{p} of $\StmPut$]
		Hierbij wordt ervoor gezorgd dat de waarde van de variabelenaam bovenop de
		stack verwijst naar de string die daaronder staat.
	\item[\smurfinline{g} of $\StmGet$]
		Zoekt de variabele op met als naam de bovenste waarde van de stack en zet
		de waarde van die variabele bovenop de stack.
	\item[\smurfinline{h} of $\StmHead$]
		Vervangt de string bovenop de stack door zijn eerste karakter.
	\item[\smurfinline{t} of $\StmTail$]
		Zet alles behalve de head van de string bovenop de stack op de stack.
	\item[\smurfinline{q} of $\StmQuotify$]
		Manipuleert de string bovenop de stack zodat die als $\StmPush$-commando in
		een Smurfprogramma kan worden gebruikt: escapet LF-karakters, dubbele
		aanhalingstekens en back\-slashes met een backslash, en plaatst dubbele
		aanhalingstekens om de hele string. Het resultaat wordt op de stack gezet.
	\item[\smurfinline{x} of $\StmExec$]
		Voert de string bovenop de stack uit als Smurfprogramma. Het programma
		begint met een schone toestand, d.w.z. met een lege stack en variable
		store. Dit commando maakt condities, recursie en iteratie mogelijk.
\end{description}


% vim: set spelllang=nl:
\subsection{Organisatie} %todo titel

In \autoref{sec:def} beschrijven we formele definities om de semantiek van
Smurf te kunnen specificeren. We kijken naar de syntax, input en output, de
programmatoestand en de transities in de natuurlijke semantiek die we gaan
definiëren. \autoref{sec:rules} beschrijft vervolgens per statement in de
syntax de formele semantiek. Hierbij baseren we ons op de Smurf specificatie
\cite{safalra}, waarbij we dingen verhelderen en ongedefinieerd gedrag
definiëren.



