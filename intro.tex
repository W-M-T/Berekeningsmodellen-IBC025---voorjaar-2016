% vim: set spelllang=nl:
\section{Inleiding}
\label{sec:intro}

Smurf is een esoterische programmeertaal oorspronkelijk ontworpen door Matthew
Westcott. In de specificatie \cite{safalra} beschrijft hij kort wat Smurf is:
\begin{quote}
	Smurf = String-based MURiel Forthoid

	Smurf is a tarpit based on the self-propagation paradigm featured in Muriel.
	The only native data type is the string, and operations are carried out on
	strings in a forty manner.
\end{quote}
We hebben dus te maken met een Forth-achtige programmeertaal. De eigenschappen
die we hiervan terugzien in Smurf zijn voornamelijk reflection,
stackgeörienteerd en `geconcateneerd programmeren'. We kunnen het programma dus
dynamisch aanpassen, werken met een stack en schrijven een programma als één
grote functiecompositie (zonder met functieapplicaties te werken). Voordat we
alle commando's bespreken is een voorbeeld op zijn plaats.

\begin{exmp}
	We bekijken het volgende programma:
	\begin{smurf}"papa" "smurf" + o\end{smurf}
	Hier gebruiken we drie functiecomposities om vier functies aan elkaar te
	knopen:
	\begin{itemize}
		\item \smurfinline{"papa"} zet de string `\texttt{papa}' op de stack.
		\item \smurfinline{"smurf"} zet de string `\texttt{smurf}' op de stack.
		\item \smurfinline{+} concateneert de twee elementen bovenop de stack
			(eerst gepushte element eerst) en zet het resultaat op de stack.
		\item \smurfinline{o} output het element bovenop de stack.
	\end{itemize}
	De output van dit programma is dus `papasmurf'.

	We hebben spaties gebruikt voor de leesbaarheid. Dit is toegestaan maar niet
	vereist. Het programma \smurfinline{"papa""smurf"+o} is eveneens geldig.
\end{exmp}

Naast de stack kent Smurf ook een \emph{variable store} die variabelenamen
(strings) naar waardes (strings) stuurt. Het gebruik hiervan is best te
illustreren met een voorbeeld:

\begin{exmp}
	We bekijken het volgende programma:
	\begin{smurf}"smurf" "papa" p "papa" g o\end{smurf}
	Nadat `smurf' en `papa' op de stack zijn gezet gebruiken we \smurfinline{p}
	om de variabele `papa' de waarde `smurf' te geven. Hierna is de stack weer
	leeg. Vervolgens zetten we `papa' op de stack en gebruiken we \smurfinline{g}
	om het bovenste element als variabele op te zoeken in de variable store en de
	waarde ervan op de stack te zetten. Hierbij wordt het bovenste element van de stack verwijdert. De stack bestaat nu dus uit het element
	`smurf'. Met \smurfinline{o} sturen we deze string naar de output.
\end{exmp}

% vim: set spelllang=nl:
\subsection{Commando's}
\label{sec:intro:commands}
We zullen nu kort op informele wijze de verschillende commando's in Smurf
beschrijven. We geven telkens de notatie in Smurf syntax en een leesbaarder
alternatief dat we hieronder zullen gebruiken. Wanneer een commando iets met
elementen op de stack doet, worden die elementen altijd verwijderd.

\begin{description}[style=nextline,font=\normalfont]
	\item[\smurfinline{"..."} of $\StmPush~\texttt{...}$]
		waarbij `\texttt{...}' een string is. Zet de string \texttt{...} op de
		stack.
	\item[\smurfinline{+} of $\StmCat$]
		Concateneert de bovenste twee strings (laagste eerst) op de stack en zet
		het resultaat op de stack.
	\item[\smurfinline{i} of $\StmInput$]
		Plaatst een string van `user input' op de stack. Hierbij wordt
		\texttt{\textbackslash} gebruikt om LF-karakters, dubbele aanhalingstekens
		en backslashes te escapen.
	\item[\smurfinline{o} of $\StmOutput$]
		Stuurt het bovenste element van de stack naar `de output'.
	\item[\smurfinline{p} of $\StmPut$]
		Zet de waarde van de variabelenaam bovenop de stack tot de string daaronder
		op de stack.
	\item[\smurfinline{h} of $\StmHead$]
		Zet het eerste karakter van de string bovenop de stack als string op de
		stack.
	\item[\smurfinline{t} of $\StmTail$]
		Zet alles behalve de head van de string bovenop de stack op de stack.
	\item[\smurfinline{q} of $\StmQuotify$]
		Manipuleert de string bovenop de stack zodat die als $\StmPush$-commando in
		een Smurfprogramma kan worden gebruikt: escapet LF-karakters, dubbele
		aanhalingstekens en backslashes met een backslash, en plaatst dubbele
		aanhalingstekens om de hele string. Het resultaat wordt op de stack gezet.
	\item[\smurfinline{x} of $\StmExec$]
		Voert de string bovenop de stack uit als Smurfprogramma. Het programma
		begint met een schone toestand, d.w.z. met een lege stack en variable
		store. Dit commando maakt condities, recursie en iteratie mogelijk.
\end{description}


% vim: set spelllang=nl:
\subsection{Voorbeelden}
\label{sec:intro:exmp}
Nu we alle commando's hebben gezien bekijken we nog een paar voorbeelden die
aangeven hoe conditionele executie kan worden behaald in Smurf.

% Dit is waarschijnlijk te veel anders.
%
%\begin{exmp}
%	Het volgende programma implementeert een simpel bestelsysteem:
%	\begin{smurf}
%		"ijs" "0" p \\
%		"pizza" "1" p \\
%		"Wil je ijs (0) of pizza (1)?\textbackslash{}n" o \\
%		"Ik ga " i g + " voor je maken!" + o
%	\end{smurf}
%	We zetten de variabelen \verb$0$ en \verb$1$ tot \verb$ijs$ en \verb$pizza$.
%	Vervolgens vragen we de gebruiker een keuze te maken. Met \smurfinline{ig}
%	kan de gebruiker in feite kiezen welke variabele wordt opgehaald.
%
%	De uitvoer kan er dus uitzien als:
%	\begin{verbatim}
%		Wil je ijs (0) of pizza (1)?
%		1
%		Ik ga pizza voor je maken!
%	\end{verbatim}
%\end{exmp}

\begin{exmp}
	\label{exmp:headortail}
	Dit programma laat de gebruiker kiezen welke functie er wordt uitgevoerd:
	\begin{smurf}
		"h""head"p "t""tail"p \\
		"Voer een string in.\textbackslash{}n"o iq \\
		"Wil je de head of de tail?\textbackslash{}n"o \\
		ig +"o"+ x
	\end{smurf}
	Allereerst zetten we de variabelen \verb$head$ en \verb$tail$ tot \verb$h$ en
	\verb$t$, de commando's voor de head en tail van een string. We vragen de
	gebruiker om een string en zetten die in quotes zodat ze gebruikt kan worden
	in een Smurfprogramma. We vragen de gebruiker om `head' of `tail' in te
	voeren en halen de bijbehorende functie op. Op dit moment kan de stack eruit
	zien als \verb$h$, \verb$"mijn string"$ (bovenste element eerst). Met
	\smurfinline{+"o"+} bouwen we hiervan het programma \smurfinline{"mijn
	string"ho}, wat we met \smurfinline{x} uitvoeren. De uitvoer van het
	programma is dus `m', in dit geval.

	Wat gebeurt er als de gebruiker iets anders dan `head' of `tail' invoert? In
	dit geval bestaat er geen variabele met de naam die de gebruiker had
	ingevoerd. Het commando \smurfinline{g} levert dan de lege string op. Dit
	betekent dat we het programma \smurfinline{"mijn string"o} bouwen. Als de
	gebruiker dus geen correcte keuze maakt, dan zal de hele string worden
	teruggegeven.
\end{exmp}

\begin{exmp}
	Het volgende programma is een Smurf interpreter:
	\begin{smurf}
		ix
	\end{smurf}
	We halen input op van de gebruiker en voeren dit uit als Smurfprogramma.
\end{exmp}

Ten slotte geven we nog een wat groter voorbeeld waar recursie wordt
voorgedaan:

\begin{exmp}
	Dit programma geeft als output de input, de tail van de input, de tail van de
	tail van de input, etc., tot en met de lege string.
	\begin{smurf}
		\footnotesize
		i "s"p \\
		"s"g"\textbackslash{}n"+o \\
		"t\textbackslash{}"s\textbackslash{}"p\textbackslash{}"s\textbackslash{}"g\textbackslash{}"\textbackslash{}\textbackslash{}n\textbackslash{}"+o\textbackslash{}"c\textbackslash{}"p\textbackslash{}"c\textbackslash{}"gq\textbackslash{}"s\textbackslash{}"gq\textbackslash{}"c\textbackslash{}"g++q\textbackslash{}"x\textbackslash{}"+\textbackslash{}"s\textbackslash{}"gp\textbackslash{}"\textbackslash{}"\textbackslash{}"\textbackslash{}"p\textbackslash{}"s\textbackslash{}"ggx" "c"p \\
		"c"gq "s"gq "c"g++ \\
		x
	\end{smurf}
	Met de eerste twee regels halen we input op en sturen die naar de output. Ook
	wordt de input in een variabele \verb$s$ opgeslagen.
	De derde regel zet een Smurfprogramma in \verb$c$.
	In de vierde regel plakken we dat gequotifiede programma, de gequotifiede
	input en het programma aan elkaar vast. Het resultaat voeren we uit.

	Op dit moment voeren we dus het programma uit dat bestaat uit twee keer een
	string pushen en vervolgens het programma uit \verb$c$. Dan bekijken we nu
	dit programma, wat we in de derde regel hierboven hebben gedefinieerd, in een
	iets makkelijkere opmaak:
	\begin{smurf}
		t "s"p "s"g "\textbackslash{}n"+o \\
		"c"p \\
		"c"gq "s"gq "c"g++ \\
		q"x"+ "s"gp \\
		""""p \\
		"s"ggx
	\end{smurf}
	Met de eerste regel slaan we de tail van de oude input op in \verb$s$. We
	sturen deze tail ook naar de output. Hierna staat alleen nog de eerste string
	op de stack, d.w.z. het programma waar we naar kijken. Dit slaan we op in
	\verb$c$.
	De derde regel hier is hetzelfde als de vierde regel hierboven: we maken een
	nieuw programma, wat er precies hetzelfde uitziet als dit, behalve dat we van
	de tweede string één karakter hebben weggehaald.

	Het is verleidelijk op dit moment weer \smurfinline{x} te gebruiken. We
	moeten echter eerst nog controleren of we niet met de lege string te maken
	hebben. Het is namelijk niet mogelijk om x uit te voeren als de stack leeg
	is. We maken dus van het nieuwe programma wederom een nieuw programma, wat
	bestaat uit het pushen van het oude programma en vervolgens \smurfinline{x}
	aanroepen. Dit programma slaan we op in de variabele met als naam de waarde
	van \verb$s$ (de input).

	Mocht deze input de lege string zijn, dan overschrijft de vijfde regel dit
	programma met een leeg programma. Vervolgens halen we in de zesde regel het
	programma (of het lege programma, in het geval van de lege string) op, en
	voeren dit uit. Door deze laatste paar regels zullen we dus niet
	\smurfinline{t} uitvoeren op de lege string.

	\medskip
	De uitvoer van dit programma met de input `smurf' is dus:

	\begin{verbatim}
		smurf
		murf
		urf
		rf
		f
	\end{verbatim}
\end{exmp}


% vim: set spelllang=nl:
\subsection{Organisatie} %todo titel

In \autoref{sec:def} beschrijven we formele definities om de semantiek van
Smurf te kunnen specificeren. We kijken naar de syntax, input en output, de
programmatoestand en de transities in de natuurlijke semantiek die we gaan
definiëren. \autoref{sec:rules} beschrijft vervolgens per statement in de
syntax de formele semantiek. Hierbij baseren we ons op de Smurf specificatie
\cite{safalra}, waarbij we dingen verhelderen en ongedefinieerd gedrag
definiëren. In \autoref{sec:anal} bekijken we een stuk code aan de hand van de
gedefinieerde regels.

\autoref{sec:planning} bevat een planning voor het afwerken van het werkstuk,
die uiteindelijk verwijderd zal worden.



